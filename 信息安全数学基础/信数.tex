\documentclass[cn,10pt]{elegantbook}
\title{信息安全数学基础}
\subtitle{密码学基础和数论基础}

\author{熊睿 \& Stander-by}
\institute{HUST-CSE-2004}
\date{Mar 7, 2022}
\version{1.0}


\extrainfo{学习新思想,争做新青年。}

\setcounter{tocdepth}{3}

\logo{cse.jpg}
\cover{cover.jpg}

% 本文档命令
\usepackage{array}
\usepackage{ulem}
\usepackage{xcolor}
\newcommand{\ccr}[1]{\makecell{{\color{#1}\rule{1cm}{1cm}}}}

\definecolor{customcolor}{RGB}{32,178,170}
\colorlet{coverlinecolor}{customcolor}

\begin{document}

\maketitle
\frontmatter

\chapter*{特别声明}

\markboth{Introduction}{前言}

该笔记是于2022年3月7日开始,针对华中科技大学网络空间安全学院2020级信息安全数学基础课程编制的笔记,该课程使用的是HUST-CSE汤学明老师所著的《信息安全数学基础》,还有中国科学院大学的陈恭亮老师所编著的《信息安全数学基础》,陈恭亮老师也是汤老师在WHU读研的导师。学有余力时,还应去翻阅华罗庚的著作《数论引导》。

该课程所运用的主要编程环境是开源的SageMath,同时根据需要可以使用python和mathematical等。

SageMath-GitHub 网址:\href{https://github.com/sagemath/}{https://github.com/sagemath/}

该课程前有《离散数学》打基础,后为《密码学原理》铺路,是理论实践之间的桥梁,理解抽象知识,提高解决实际数学、密码学问题的能力。

同时使用优美的\LaTeX{}来书写,也少不了WHU好友的鼎力推荐(Thanks),也希望能通过这次机会,去熟悉\LaTeX{}的语法,特别是数学公式的编写,和作图能力。

\vskip 0.5cm

另外,“多分享,多奉献”,之后我会将文档同步于我的Github账号,由于编者的水平有限,会存在一些难免的疏漏和问题,请大家批评指正!

\vskip 0.5cm

我的GitHub 网址:\href{https://github.com/Stander-by/}{https://github.com/Stander-by/}

我的邮件:\email{xiongruistanderby@gmail.com}

NodeBB校内网: \href{http://10.12.162.1:5881/}{http://10.12.162.1:5881/}

NodeBB校外网: \href{http://124.71.166.97:5881/}{http://124.71.166.97:5881/}

SageMath:\href{http://10.12.162.1:5883/}{http://10.12.162.1:5883/}
\vskip 1.5cm

\begin{flushright}
Stander-by\\
Mar 7, 2022
\end{flushright}

\tableofcontents

\mainmatter

\chapter{整除}
第一章讲述整数整除中带余除法、因子、最大公因数、最小公倍数和算术基本定理等基本概念,希望通过从带余除法到最终的算术基本定理的推导过程,掌握整数研究的一般方法,在第三章多项式的唯一因式分解定理的推到过程中,用到这种系统化的方法。
\section{整除性}

\begin{theorem}[带余除法]\label{thm1.1}
  任意给定整数a和整数b>0,存在唯一的一对整数q,$0 \leq r \textless b$,使得:$a = qb + r$。
\end{theorem}

\begin{corollary}
  任意给定整数a和整数b<0,存在唯一的一对整数q,$0\leq r < \left| b \right| $,使得:$a = qb + r$。
\end{corollary}

\begin{corollary}
任意给定整数a,c和整数b$\ne$0,存在唯一的一对整数q,$c\leq r < \left| b \right| + c$,使得$a = qb + r$。
\end{corollary}

\begin{theorem}[整除性质]\label{thm1.2}
设a,b,c为整数
\begin{enumerate}[(1)]
\item 若$a \mid b $,$ b \mid a $,则$ a = \pm b $;
\item 设整数$k \neq 0$,若$a\mid b$,则$\pm ka \mid \pm kb$,反之亦然;
\item 对任意整数k,若$a\mid b $,则$a \mid kb$;
\item 若$a \mid b$,$b \neq 0$,则$\frac{b}{a} \mid b$;
\item 若$a \mid b$,$b \mid c$,则$a \mid c$;
\item 若$a \mid b$,$a \mid c$,则对任意整数s和t,$a \mid sb+ tc$;  
\end{enumerate} 
\end{theorem}

\begin{example}
  若a是整数,证明$a^3-a$是6的倍数。
\end{example}

\begin{proof}
\\$a^3-a=(a-1)a(a+1)$
\\由定理 \ref{thm1.2}之(3)有,$3 \mid (a-1)a(a+1)$,可设$(a-1)a(a+1) = 3q$。
\\同理,因为两个连续的整数中一定有一个是2的倍数,所以$2 \mid (a-1)a(a+1)$,不妨设$q = 2t$,于是$(a-1)a(a+1)= 6t$,根据整除的定义,$6 \mid (a-1)a(a+1)$。
\end{proof}

\begin{example}
  若x,y为整数,证明:$10 \mid 2x+5y \Leftrightarrow 10 \mid 4x+5y$。
\end{example}
\begin{proof}
\\由定理 \ref{thm1.2}之(4)有
\\$4x+5y = 10(x+2y)-3(2x+5y)$
\\$2x+5y = 3(4x+5y)-10(x+y)$
\end{proof}
\begin{exercise}
  \begin{enumerate}[1)]
    \item 是否存在n,使得$n+1 \mid n!$?
    \item 如果$n \mid x^2 -y^2$,那么$n \mid x+y$或者$n \mid x-y$是否成立?
  \end{enumerate}
\end{exercise}
\begin{note}

\end{note}

\section{欧几里得辗转相除法}
\begin{definition}
  $(a,b) \ gcd(a,b)$表示为a和b的最大公因数。
\end{definition}
\begin{theorem}
  设a,b是两个不全为0的整数,且$a = qb+r $,r为整数,则$(a,b) = (b,r)$。\\
\end{theorem}

\begin{proof}
  \\设$d = (a,b) \ d^{\prime} = (b,r)$
  \\由定理 \ref{thm1.2} (6)得,$d \mid (a-qb)$即$d \mid r$
  \\且$d \leq d^{\prime}$,所以$d = d^{\prime}$。
\end{proof}

\begin{corollary}
  设a,b是两个不全为0的整数,q为整数,则$(a,b)=(a\pm bq,b) = (a, b \pm aq)$。
\end{corollary}

\begin{example}
  证明:若n为整数,则$(21n+4,14n+3) = 1$,$(n^3+2n,n^4+3n^2+1) = 1$。
\end{example}

\begin{theorem}[辗转相除法]
  设a,b是两个正整数,下式为起欧几里得辗转相除算式
  \\令$r_0 = a \quad r_1 = b $,反复运用带余除法算式:
  \begin{align*}
    r_0 =& r_1q_1+r_2  && 0 \leq r_2<r_1
    \\r_1 =& r_2q_1+r_3  && 0 \leq r_3<r_2
    \\&\vdots &&\vdots
    \\r_{n-2} =& r_{n-1}q_{n-1}+r_n  && 0 \leq r_n<r_{n-1}
    \\r_{n-1} =& r_{n}q_{n}+r_{n+1}  && r_{n+1} = 0
  \end{align*}
  由于$r_{n+1}<r_n<r_{n-1}<\dot <r_2<r_1 = b$,且b为有限正整数,所以经过有限的正整数,所以经过有限的步骤,必然存在n,使得$r_{n+1} = 0$。
  \begin{enumerate}[(1)]
    \item $(a,b)=r_n$
    \item 存在整数s,t,使得$r_n=sa+tb$
    \item 任意整数c,若满足$c \mid a$且$c \mid b$,则$c \mid r_n$
  \end{enumerate}
\end{theorem}

\begin{theorem}{递推法求线性组合}
  设a,b是两个正整数,$q_i$为其欧几里得辗转相除算式的部分商,则由
  \begin{align*}
    S_0 &= 0
    \\S_1 &= 1
    \\S_{i+1} &=S_{i-1}-q_{n-i}S_i \quad i \geq 1
  \end{align*}
  所得的$S_{n-1}$和$S_n$满足$S_{n-1}a+S_n b = r_n$。
  \begin{align*}
    \begin{pmatrix}
      r_1\\
      r_2
    \end{pmatrix}& =\begin{pmatrix}
      0 & 1\\
      1 & -q_1
    \end{pmatrix}\begin{pmatrix}
      r_0 \\
      r_1
    \end{pmatrix}\\
    \begin{pmatrix}
      r_2\\
      r_3
    \end{pmatrix}& = \begin{pmatrix}
      0 & 1\\
      1 & -q_2
    \end{pmatrix}\begin{pmatrix}
      r_1\\
      r_2
    \end{pmatrix}\\
    &\dots  \dots \\
    \begin{pmatrix}
      r_{n-1}\\
      r_n
    \end{pmatrix}& = \begin{pmatrix}
      0 & 1\\
      1 & -q_{n-1}
    \end{pmatrix}\begin{pmatrix}
      r_{n-2}\\
      r_{n-1}
    \end{pmatrix}
  \end{align*}
\end{theorem}

\begin{example}
  试求s,t,使得$30111s+4520t = (30111,4520)$。
\end{example}
\begin{solution}
  \begin{tabular}{|c|c|c|c|c|c|c|c|c|c|c|c|c|}
    \hline i&0&1&2&3&4&5&6&7&8&9&10&11\\
    \hline $q_{12-i}$& & &2&1&1&4&1&21&1&1&1&6\\
    \hline $S_i$&0&1&-2&3&-5&23&-28&611&-639&1250&-1889&12584\\
    \hline
  \end{tabular}
\end{solution}
\begin{note}
  $S_{i+1}=S_{i-1}-q_{n-i}S_i$
\end{note}
\begin{exercise}
  \begin{enumerate}[1)]
    \item 如果$(a,b) = d$,那么$\{ sa+tb \mid s , t \in Z \} $是什么?
    \item 试求出一对整数s和t满足$s*12345 - t*345 = (12345,345)$。
  \end{enumerate}
\end{exercise}

\section{一次不定方程}
\begin{theorem}
  设a,b是两个不全为0的整数,则
  \begin{enumerate}[(1)]
    \item 对于任何正整数$k$,$(ka,kb)=k(a,b)$
    \item $(\frac{a}{(a,b)} ,\frac{b}{(a,b)} ) = 1$
  \end{enumerate}
\end{theorem}
\begin{proof}
  (2)由(1),$(a,b)(\frac{a}{(a,b)} , \frac{b}{(a,b)})=((a,b)\frac{a}{(a,b)},(a,b)\frac{b}{(a,b)}) = (a,b)$\\
  因为$(a,b) \neq 0$,所以$(\frac{a}{(a,b)} , \frac{b}{(a,b)}) = 1$。
\end{proof}

\begin{theorem}
  设a,b,c为整数,$a \neq 0,c \neq 0$,若$(a,b) = 1$,则$(a,bc) = (a,c)$。
\end{theorem}
\begin{remark}
  $d = (a,bc),d^{\prime} = (a,c) \quad  proof d \mid d^{\prime} \ and \ d^{\prime} \mid d$。 
\end{remark}

\begin{corollary}
  设a,b,c为整数,$a \neq 0$,若$(a,b) = 1$,$a \mid bc $,则$a \mid c$。
\end{corollary}

\begin{example}
  证明:若$a_1b_1-a_2b_2 = 1$,则$(a_1+a_2,b_1+b_2) = 1$。
\end{example}
\begin{proof}
  $a_1b_1 - a_2b_2 = 1 \Rightarrow (b_1,b_2) = 1,(a_1,a_2)=1 $ comes from \  \ref{thm1.2}(6)\\
  $(b_1,b_1+b_2) = (b_1,b_2) = 1$\\
  $(b_1+b_2,b_1(a_1+a_2)) = (b_1+b_2,a_1+a_2)$\\
  $(b_1+b_2,b_1(a_1+a_2)) = (b_1+b_2,b_1(a_1+a_2)-a_2(b_1+b_2)) = (b_1+b_2,1) = 1$。
\end{proof}

\begin{theorem}
  设a,b是两个完全不为0的整数,整系数不定方程$ax+by = c$有解的充分条件是$(a,b) = c$。此时,若$x = x_0,y=y_0$是方程的一个特解,那么方程的所有整数解可以表示为:\\
  $$
  \left\{
    \begin{aligned}
      x &= x_0- \frac{b}{(a,b)}t  \\
      y &= y_0+ \frac{a}{(a,b)}t  \\
      t & \in Z
    \end{aligned}
  \right.
  $$
\end{theorem}

\begin{example}
  求不定方程$18x+7y = 44$的所有整数解
\end{example}
\begin{solution}
  根据辗转相除法可得\\
  $18*2 + 7*(-5) = 1$
  特解为:
  $$
  \left\{
    \begin{aligned}
      x &=2\\
      y &=-5
    \end{aligned}
  \right.
  $$
  不定方程$18x+7y = 44$的特解为:
  $$
  \left\{
    \begin{aligned}
      x_0 &=2 \times 44 = 88\\
      y_0 &=(-5) \times 44 = -220
    \end{aligned}
  \right.
  $$
  原不定方程的特解为:
  $$
  \left\{
    \begin{aligned}
      x &= 88 - 7t\\
      y &= -220+18t\\
      t & \in Z
    \end{aligned}
  \right.
  $$
\end{solution}

\begin{example}
  \\
  1.求不定方程$a_1x_1+a_2x_2+ \dot +a_{n-1}x_{n-1}+x_n = N$的所有整数解。\\
  2.求不定方程$18x+7y+6z = 44$的所有整数解。\\
  3.求不定方程$6x+10y+15z = 44$的所有整数解。
\end{example}
\begin{solution}
  \\2.$(18,7) = 1 \Rightarrow 18x+7y$可以表示任何整数,$z$可以任取。\\
  转化为$18x+7y = 44 -6z$来求得二元不定方程的解。\\
  3.$(6,10) =2 \Rightarrow 6x +10y = 2t$\\
  原方程转化为$2t+15z = 44$\\
  其解为:
  $$
  \left\{
    \begin{aligned}
      t &=-308-15t_1\\
      z &= 44+2t_1\\
      t_1 & \in Z
    \end{aligned}
  \right.
  $$
  $3x+5y = t$的所有整数解可以表示为;
  $$
  \left\{
    \begin{aligned}
      x &=2t-5t_2\\
      y &=-t+3t_2\\
      t_2 & \in Z
    \end{aligned}
  \right.
  $$
\end{solution}
\begin{exercise}
  求解不定方程$x^2+y^2=z^2$
\end{exercise}

\section{最小公倍数}
\begin{definition}
  设a,b是两个不全为0的整数,整数满足$a \mid c, b \mid c$,则称c为a和b的公倍数,在a和b的所有公倍数中,一定有一个正的最小公倍数,称为a和b的最小公倍数,记作$[a,b]$,或者$lcm(a,b)$。
\end{definition}

\begin{theorem}
  设a,b是两个正整数,且$(a,b) = 1$,
  \begin{enumerate}[(1)]
    \item 若$a \mid c,b \mid c$,则$ab \mid c$
    \item $[a,b] = ab$
  \end{enumerate}
\end{theorem}

\begin{theorem}
  设a,b是两个正整数,
  \begin{enumerate}[(1)]
    \item 对于任何正整数k,$[ka,kb] = k [a,b]$
    \item $[a,b]= \frac{ab}{(a,b)}$
    \item 若$a \mid c, b \mid c$,则$[a,b] \mid c$
  \end{enumerate}
\end{theorem}

\begin{example}
  若$x,y, \sqrt{x} + \sqrt{y} $均为整数,试证明$\sqrt{x} , \sqrt{y}$均为整数。
\end{example}

\begin{proof}
  若$\sqrt{x}+\sqrt{y}=0$,则$x=y=0$,结论成立\\
  $\sqrt{x} = \frac{1}{2} ((\sqrt{x}+\sqrt{y})+(\sqrt{x}-\sqrt{y}))$所有$\sqrt{x},\sqrt{y}$都是有理数\\
  不妨设$\sqrt{x}=\frac{b}{a}$,a,b为正整数,且$(a,b)=1$\\
  $x =  \frac{b^2}{a^2}$为整数,$(a^2,b^2)=a^2 \Rightarrow (a,b)=(a,b^2) = (a^2,b^2)$\\
  所以$a= 1,\  \sqrt{x}$为整数。
\end{proof}

\begin{theorem}
  $a_1,a_2,\dots,a_n $不全为$0$的整数,不妨设$a_1 \neq 0$,定义$d_1 = (a_1,a_2),d_2=(d_1,a_3),\dots ,d_{n-1}=(d_{n-2},a_n)$,则$(a_1,a_2,\dots,a_n) = d_{n-1}$。
\end{theorem}

\begin{conclusion}
  若正整数$d = (a_1,a_2,\dots,a_n)$,则存在整数$s_1,s_2,\dots,s_n$,使得$d=s_1a_1+s_2a_2+\dots +s_na_n$
\end{conclusion}

\begin{theorem}
  正整数c是$a_1,a_2,\dots,a_n$的最大公因数,当且仅当:
  \begin{enumerate}[(1)]
    \item $c \mid a_1, c \mid a_2, \dots ,c \mid a_n$。
    \item 任何整数$c^{\prime}$若满足$c^{\prime} \mid a_1,c^{\prime} \mid a_2, \dots , c^{\prime} \mid a_n$,则$c^{\prime} \mid c$。
  \end{enumerate}
\end{theorem}

\begin{theorem}
  设$a_1,a_2,\dots , a_n$是n个不为0的整数,定义$m_1 = [a_1,a_2],m2 = [m_1,a_3],\dots , m_{n-1} = [m_{n-2},a_n]$,则$[a_1,a_2,\dots , a_n]= m_{n-1}$。
\end{theorem}
\begin{theorem}
  正整数m是$a_1,a_2,\dots,a_n$的最大公因数,当且仅当:
  \begin{enumerate}[(1)]
    \item $ a_1 \mid m, a_2 \mid m, \dots ,a_n \mid m$。
    \item 任何整数$m^{\prime}$若满足$a_1 \mid m^{\prime}, a_2 \mid m^{\prime}, \dots ,a_n \mid m^{\prime}$,则$m \mid m^{\prime}$。
  \end{enumerate}
\end{theorem}

\section{素数与算术基本定理}
\begin{definition}
  设p是一个整数,$p \neq 0, \pm 1$,如果除了$\pm 1, \pm p$外,p没有其他因数,则称p为素数(或者质数,不可约数),否则为合数(可约数),最小的素数为2。
\end{definition}
\begin{theorem}
  合数m的最小不等于1的正因子p一定是素数,且$p \leq \sqrt{m}$。
\end{theorem}
\begin{proof}
  反证法证明$p$一定是素数。\\
  由 $ \ref{thm1.2}\ (4) \Rightarrow \frac{m}{p} \mid m$,又$1<\frac{m}{p} <m$,$p \leq \frac{m}{p} \Rightarrow p \leq \sqrt{m}$。
\end{proof}

\begin{conclusion}
  设整数$m>1$,如果所有不大于$\sqrt{m}$的素数都不是m的因子,那么m是素数。\\
  整数$m>1$是合数的充要条件是存在不大于$\sqrt{m}$的素因子。
\end{conclusion}

\begin{theorem}
  素数有无穷多个。
\end{theorem}
\begin{proof}
  反证法证明,有有限个素数$p_1,p_2,\dots , p_n$\\
  考虑整数$A = p_1p_2 \dots p_n +1$\\
  不妨设$p_i \mid A \Rightarrow p_i \mid A -p_1p_2\dots p_n$,即$p_i \mid 1$,矛盾。
\end{proof}

\begin{theorem}[素数定理]
  $\pi (x)$表示不超过x的素数个数。
  \begin{align*}
    \lim_{x \rightarrow \infty} \pi (x) \frac{ln(x)}{x} = 1
  \end{align*}
\end{theorem}
\begin{theorem}[伯特兰-切比雪夫定理]
  设整数$n>3$,至少存在一个素数p满足$n<p<2n-2$。
\end{theorem}
\begin{theorem}[算术基本定理]
  设n是一个大于1的正整数,那么n一定可以分解成一些素数的乘积。若规定n的所有素因子按照从小到大的顺序排列,那么n的分解方式是唯一的。
\end{theorem}
\begin{definition}
  设n是大于1的正整数,$n = \prod_{i=1}^{s} p_i^{\alpha_i}(\alpha_i>0,p_i<p_j(i<j))$称为n的标准分解式。
\end{definition}

\begin{example}
  设$m = \prod_{i=1}^{s}p_i^{\alpha_i},\ n = \prod_{i=1}^{s}p_i^{\beta_i},\alpha_i \geq 0 , \beta_i \geq 0, \ p_i \neq p_j(i \neq j)$ proof:\\
  \begin{enumerate}[(1)]
    \item $m \mid n \Leftrightarrow 1 \leq i \leq s , \alpha_i \leq \beta_i$\\
    \item $(m,n) = \prod_{i=1}^{s}p_i^{\min{\alpha_i,\beta_i}}$\\
    \item $[m,n] = \prod_{i=1}^{s}p_i^{\max{\alpha_i,\beta_i}}$\\
  \end{enumerate}
\end{example}
\begin{exercise}
  证明形如$4k+3$的素数有无穷多个。
\end{exercise}
\begin{proof}
  反证法证明,假设形如$4k+3$素数仅有有限个,其全部为$p_1,p_2, \dots ,p_n$,均为大于1的正整数,考虑整奇数:
  $A = 4p_1p_2 \dots p_n -1$\\
  A是一个形如$4k+3$的合数\\
  所以A的素因子全部都是形如$4k+1$的素数,但形如$4k+1$的整数相乘仍然是形如$4k+1$的整数,矛盾。
\end{proof}
\begin{exercise}
  \begin{enumerate}
    \item 试证明:$\max\{a,b,c\} = a+b+c+ \min \{a,b,c\}-\min \{a,b\} - \min \{b,c\} - \min \{a,c\}$
    \item 试证明:$gcd(a,lcm(b,c))=lcm(gcd(a,b),gcd(a,c))$
    \item 试用算术基本定理证明:
    \begin{align*}
      \prod_{p} \frac{p}{p-1} = \sum_{i=1}^{+ \infty} \frac{1}{i}
    \end{align*}
  \end{enumerate}
\end{exercise}

\section{高斯函数}
\begin{definition}
  实数x的 \textbf{高斯函数} $[x]$是指不超过x的最大的整数,$[x]$也称为x的 \textbf{整数部分} ,x的\textbf{小数部分} ${x}$是指$x-[x]$。
\end{definition}

\begin{theorem}
  对于$[x]$,以下结论正确:
  \begin{enumerate}
    \item 若$x \leq y \Rightarrow [x] \leq [y]$
    \item 整数a满足$x-1<a \leq a+1 \Leftrightarrow a = [x]$
    \item 整数a满足$a \leq x <a+1 \Leftrightarrow a= [x]$
    \item 对于任意整数n,$[n+x]=n+[x]$
  \end{enumerate}
\end{theorem}
\begin{proof}
  (4)\\$n+x-1<[n+x] \leq n+x$
  \\$x-1<[n+x]-n \leq x$
  \\$[n+x]-n = [x]$
\end{proof}
\begin{example}
  对于任意实数x,y,试证明$[x+y] \geq [x]+[y]$
\end{example}
\begin{proof}
  \\
  $[x+y] = [[x]+[y]+\{x\}+\{y\}]$\\
  $[x+y] = [x]+[y]+[\{x\}+\{y\}]$\\
  $[\{x\}+\{y\}] \geq 0$\\
  $[x+y] \geq [x]+[y]$
\end{proof}
\begin{theorem}
  对于整数a,b,且$b>0$,带余除法算式为$a = qb +r, 0 \leq r <b$,则$q = [ \frac{a}{b}]$。
\end{theorem}
\begin{proof}
  \\$q = \frac{a}{b}- \frac{r}{b},\quad 0 \leq \frac{r}{b} <1$\\
  \\$\frac{a}{b} -1 \leq q = \frac{a}{b}- \frac{r}{b} < \frac{a}{b}$\\
\end{proof}
\begin{theorem}
  设$p$是一个素数,则$n!$中包含$p$的幂次为$\sum_{i>1}[\frac{n}{p^i}]$。
\end{theorem}
for example:
 \[
  30!= 1 \times 2 \times \cdots \times 30 = p_1^{e_1}p_2^{e_2} \cdots = 2^{e_1}3^{e_2} \cdots 
 \]
 \[
  [\frac{30}{2}]+[\frac{30}{2^2}] +[\frac{30}{2^3}] +[\frac{30}{2^4}]+[\frac{30}{2^5}] \cdots 
 \]
 \[
   =15+7+3+1+0+ \cdots
 \]
所以$e1 = 15+7+3+1$
\begin{proof}
  定理1.22\\
  \\$p^i$的倍数共有$[\frac{n}{p^i}]$\\
  \\当$(p,k) = 1$时,整数$p^ik$为$n!$提供的p的幂次为i\\
  \\由于$p^ik$同时是$p,p^2,\dots ,p^i$的倍数,所以恰好共计数了$i$次\\
  \\因此$n!$中包含$p$的幂次为$\sum_{i>1}[\frac{n}{p^i}]$\\
\end{proof}
\begin{example}
  试证明$ \binom{100}{50} $,的十进制末位数不为0.
\end{example}
\begin{proof}
  $\\ \binom{100}{50} = \frac{100!}{50!50!}$\\
  $\\100! = 2^{e_1}3^{e_2}5^{e_3} \cdots$\\
  $\\50! =  2^{b_1}3^{b_2}5^{b_3} \cdots$\\
  $\\b_3 = [\frac{50}{5}]+[\frac{50}{25}]+[\frac{50}{125}] $\\
  $\\e_3 = [\frac{100}{5}]+[\frac{100}{25}]+[\frac{100}{125}]$\\
  \\所以$\binom{100}{50}$没有素因数5,所以末位不为0.\\
\end{proof}
\begin{example}
  设n,m为正整数,$n>m$,试证明:$\binom{n}{m} = \frac{n(n-1)\dots(n-m+1)}{m!}$是整数。
\end{example}
\begin{proof}
  \\ $\binom{n}{m} = \frac{n!}{m!(n-m)!}$\\
  \\ 只需证明$\sum_{i \geq 1}[\frac{n}{p_i}] \geq \sum_{i \geq 1}[\frac{m}{p_i}]+ \sum_{i \geq 1}[\frac{n-m}{p_i}]$\\
  \\由例题1.10可知$[\frac{n}{p_i}] \geq [\frac{m}{p_i}]+ [\frac{n-m}{p_i}]$\\
\end{proof}

\begin{problemset}
 \item 设a,m,n均为正整数,试着证明$(a^m-1,a^n-1) = a^{(m,n)}-1$
 \begin{proof}
   $Might\ as\ well,m \geq n,m = nq+r, 0 \leq r<n$ 
   \\$a^m -1 = a^{nq+r}-a^r+a^r-1 = a^r(a^{nq} -1)+a^r-1$
   \\$ = a^r(a^n-1)(1+a^n+a^{2n}+ \cdots +a^{(q-1)n})+a^r-1$
   \\so $(a^m-1) mod (a^n-1) = a^r-1$
   \\$using\ division\  algorithm,\ the \ answer \ is \ a^{(m,n)}-1.$
 \end{proof}
 \item $The \ Fermat \  number \ is \  F_n = 2^{2^n}+1,if \ F_n \ is \ prime \ number\ ,try \ to \ proof: \\ if \ 2^m+1 \ is \ prime \ number\, then\ m \ just \ like \ 2^n \in Z.$
 \begin{proof}
   
 \end{proof}
\end{problemset}
\chapter{同余}
\section{同余的基本性质}
\begin{definition}
  设m是正数,a,b为两个整数, \uline{如果a-b是m的倍数},那么称a和b关于m同余,记作$a \equiv b(mod\ m)$,否则称a和b关于m不同余,记作$a \not\equiv b(mod \ m )$。
\end{definition}
\begin{theorem}
  同余式等价关系
  \begin{enumerate}
    \item 自反性
    \item 对称性
    \item 传递性
  \end{enumerate}
\end{theorem}
\begin{definition}
  设m是正整数,全体整数按照模m同余可以划分成m个不同的等价类,称为模m剩余类,整数a所在的剩余类记为$\bar{a}$,整数x属于剩余类$\bar{a}$当且仅当$x \equiv a(mod \ m)$。从每个剩余类中取出一个整数形成的m元素称为模m完全剩余系。
\end{definition}
例如,若m=6,则模6剩余类共有6个
可以表示为$\bar{0},\bar{1},\bar{2},\cdots ,\bar{5} $
而${0,1,2,3,4,5},{-6,7,2,3,4,5}$都是模6完全剩余系。
\begin{definition}
  设m是正整数,如果整数a与m互素,那么a所在的剩余类$\bar{a}$称为模m简化剩余类,从每个简化剩余类中取出一个整数形成的集合称为模m的简化剩余系。在整数1,2,$\cdots $,m中所有与m互素的整数的个数称为m的欧拉函数,记作$ \varphi (m)$,简化剩余系共有$ \varphi (m)$个整数。
\end{definition}
\begin{example}
  设m,n为正整数,试将模m的剩余类$\bar{a}$拆分成模mn的剩余类的和。
\end{example}
\begin{solution}
  \begin{itemize}
    \item 模m剩余类$a(mod \ m)$可以表示为$\{a+mk|k \in Z\}$,而全体整数按照模n又可以分成n个剩余类,即$\bar{0},\bar{1},\cdots,\bar{n-1}$。
    \item $\{ a+mk | k \in Z\}=\{a+m(tn)|t \in Z\} \cup \{a+m(tn+1)|t \in Z\} \cup \cdots \cup \{a+m(tn+n-1)| t \in Z\} $。
    \item $ = \{a+mnk|t \in Z\} \cup \{a+m+mnt| t \in Z\} \cup \cdots \cup \{a+m(n-1)+mnt | t \in Z\}$
    \item 也就是,模m的剩余类$\overline{a}$可以拆分成n个模mn的剩余类,$\overline{a} ,\overline{a+m} ,\cdots,\overline{a+m(n-1)} $。
  \end{itemize}
\end{solution}
\begin{theorem}[同余的性质]
  设m,n是正整数,$a \equiv b(mod \ mn)$,则$a \equiv b(mod \ m),a \equiv b(mod \ n)$。
\end{theorem}
\begin{note}
  定理2.2的逆定理不成立
\end{note}
\begin{theorem}
  设m,n是正整数,若$a \equiv b(mod \ n),a \equiv b(mod \ n)$,则$a \equiv b(mod[m,n])$。
\end{theorem}
\begin{theorem}
  关于同余,以下性质成立。
  \begin{enumerate}[(1)]
    \item 若$a \equiv b(mod \ m) \Rightarrow a+c =b+c(mod \ m) $。
    \item 若$a \equiv b(mod \ m),k \in Z \Rightarrow ak \equiv bk(mod \ m)$。
    \item 若$ak \equiv bk(mod \ m),k \in Z,(k,m) =1  \Rightarrow a \equiv b(mod \ m)$。
    \item 若$a \equiv b(mod \ m),k \in Z^{+} \Leftrightarrow ak \equiv bk(mod \ mk)$。
    \item 若$a \equiv b(mod \ m)$ ,$f(x)$ 为任一整系数多项式,则$f(a) \equiv f(b)(mod \ m)$。
  \end{enumerate}
\end{theorem}
\begin{conclusion}
  若$a_1 \equiv a_2 (mod \ m),b_1 \equiv b_2 (mod \ m) \Rightarrow a_1+b_1 \equiv a_2+b_2(mod \ m), a_1b_1 \equiv a_2b_2(mod \ m)$。
\end{conclusion}
\begin{example}
  试证明正整数m能被3整除的充要条件是它的十进制表示各数位上数字之和是3的倍数。
\end{example}
\begin{proof}
  $m = (a_{n-1} \cdots a_1a_0)_{10}$\\ \\
  $\sum_{i=0}^{n-1} a_i10^i \equiv \sum_{i=0}^{n-1}a_i1^i$\\ \\
  $m(mod \ 3) = \sum_{i=0}^{n-1}a_i(mod \ 3)$
\end{proof}
\begin{note}
  讨论9,7,11,1001,13 \\ 
  \begin{itemize}
    \item $\sum_{i=0}^{n-1} a_i10^i \equiv \sum_{i=0}^{n-1}a_i1^i = \sum_{i=0}^{n-1}a_i(mod \ 9)$\\
    \item $\sum_{i=0}^{n-1} a_i10^i \equiv \sum_{i=0}^{n-1}a_i(-1)^i (mod \ 11)$\\
    \item $12345678 = 12 \times {(10^3)}^2+345 \times 10^3 +678\\=12 \times (-1)^2+345 \times (-1)+678(mod \ 1001)$\\
    \item $1001 = 7 \times 11 \times 13$
  \end{itemize}
\end{note}
\begin{exercise}
  \begin{enumerate}
    \item 若x,y为整数,证明$10|2x+5y \Leftrightarrow 10 | 4x+5y$。
    \begin{solution}
      \\
      $10|2x+5y \Leftrightarrow 2|2x+5y, 5|2x+5y.$
     \[
       \begin{cases}
         2x+5y \equiv 0(mod \ 2)\\
         2x+5y \equiv 0(mod \ 5)
       \end{cases}
       \Leftrightarrow \quad 
       \begin{cases}
         y \equiv 0(mod \ 2)\\
         2x \equiv 0(mod \ 5)
       \end{cases}
       (2,5)=1 \quad \Leftrightarrow
       \begin{cases}
        y \equiv 0(mod \ 2)\\
        x \equiv 0(mod \ 5)
       \end{cases}
      \]
      \[
       \begin{cases}
         4x +5y \equiv 0(mod \ 2)\\
         4x+5y \equiv 0(mod \ 5)
       \end{cases}
       \Leftrightarrow
       \begin{cases}
         x \equiv 0(mod \ 2)\\
         y \equiv 0(mod \ 5)
       \end{cases}
     \]
    \end{solution}
    \item 计算$2^{1024}( mod \ 10240 )$
    \begin{solution}
      \\考虑$2^{1024} \equiv x(mod \ 10240) \Rightarrow 2^{1013} \equiv y(mod \ 5)$
      \begin{align*}
        2^{1013}&=2 \times 2^{1012}\\
        &=2(2^2)^{506}\\
        &\equiv 2(-1)^{506}
      \end{align*}
      所以$2^{1013}\equiv 2 (mod \ 5)$
    \end{solution}
  \end{enumerate}
\end{exercise}
\section{欧拉函数}
\subsection{剩余系的遍历}
\begin{theorem}
  设m是正整数,若$(a,m) =1 $,则当x遍历模m的一个完全剩余系时,对于任意整数b,$ax+b$遍历模m的一个完全剩余系;当x遍历模m的一个简化剩余系,ax遍历模m的一个简化剩余系。
\end{theorem}
\begin{proof}
  m的剩余系为$\{ x_1,x_2,\cdots ,x_m\}$\\
  不妨设$1 \leq i \leq j \leq m$\\
  $ax_i+b \equiv ax_j+b(mod \ m) \Rightarrow x_i \equiv x_j(mod \ m)$\\
  矛盾\\
  $ax_i$与m互素,且两两互不同余。
\end{proof}
\begin{theorem}
  设m,n为正整数,$(m,n) =1$,则当x遍历模m的一个完全剩余系,y遍历模m的一个完全剩余系时,$mx+ny$遍历模$mn$的一个完全剩余系;当x遍历模m的一个简化剩余系,y遍历模m的一个简化剩余系时。$mx+ny$遍历$mn$的一个简化剩余系。
\end{theorem}
\begin{proof}
  $(x_i,y_j)$一共有mn种取法\\
  $i = j$取法相同,$i \neq j$取法不同\\
  $mx_i+ny_j \equiv mx_{i^{\prime}}+ny_{j^{\prime}}(mod \ mn)$\\
  $\Rightarrow y_j \equiv y_{j^{\prime}}(mod \ m)$\\
  $proof: (x_i,n)=1,(y_j,m)=1 \Rightarrow (mx_i+ny_j,mn) =1$\\
  $(mx_i+ny_j,m) = 1,(mx_i+ny_j,n) =1 $\\
  $\varphi(m) \varphi(n)$个元素和mn互素\\
  $\varphi(m) \varphi(n) \leq \varphi(mn)$\\
  证明$\varphi(mn)$无遗漏\\
  当x遍历完全,y遍历完全,$mx+ny$遍历完全\\
  $\forall a \in Z, \exists x_i,y_j \Rightarrow a \equiv a \equiv mx_i+ny_j(mod \ mn)$\\
  \[
    (a,mn) = 1 \Rightarrow
    \begin{cases}
      (a,m) =1\\
      (a,n) =1
    \end{cases}
    \Rightarrow
    \begin{cases}
      (mx_i+ny_j,m)=1\\
      (mx_i+ny_j,n)=1
    \end{cases}
    \Rightarrow
    \begin{cases}
      (y_j,m)=1\\
      (x_i,n)=1
    \end{cases}
  \]
\end{proof}
\subsection{欧拉函数}
\begin{theorem}
  设$m,n$为正整数,若$(m,n)=1$,则$\varphi(mn) = \varphi(m) \varphi(n)$[积性函数]。
\end{theorem}
\begin{proof}
  \\由定理2.6,当x遍历模n的一个简化剩余系,y遍历模m的一个简化剩余系时,mx+ny遍历模mn的一个简化剩余系,所以模mn的一个简化剩余系中的元素个数为$\varphi(mn) = \varphi(m) \varphi (n)$。
\end{proof}
\begin{theorem}
  设p为素数,e为正整数,则$\varphi(p^e) = p^e-p^{e-1}$。
\end{theorem}
\begin{proof}
  \\p是素数$\varphi(p) = p-1(1,2,\cdots,p-1)$\\
  $\varphi(p^e)$不互素的有$p,p^2,\cdots,p^{e-1},p^e$\\
  $\varphi(p^e) = p^e-p^{e-1}$
\end{proof}
\begin{theorem}
  设m为正整数,其标准分解式为$m = \prod_{i=0}^s p_i^{\alpha_i}$,则$\varphi(m)=m\prod_{i=0}^s (1-\frac{1}{p_i})$
\end{theorem}
\begin{proof}
  由定理2.7和定理2.8,\\  \\
  $\varphi(m) = \prod_{i=1}^s\varphi(p_i^{\alpha_i}) = \prod_{i=1}^s(p_i^{\alpha_i}-p_i^{\alpha_i-1})=\prod_{i=1}^sp_i^{\alpha_i}(1-\frac{1}{p_i})=m\prod_{i=1}^s(1-\frac{1}{p_i})$。\\
\end{proof}
\begin{example}
  试求$\varphi(2^33^27)$
\end{example}
\begin{solution}
  \\
  $\varphi(2^33^27)=2^33^27 \times (1-\frac{1}{2})(1-\frac{1}{3})(1-\frac{1}{7})=144$\\
\end{solution}
\begin{example}
  假设m是两个不相等的素数的乘积,如果已知$\varphi(m)=a$,试求这两个素数。
\end{example}
\begin{solution}
  \\
  由题意可设$m = pq,\ p,\ q$为不相等的素数,可得二元二次方程组,
  \[
    \begin{cases}
      m = pq\\
      (p-1)(q-1) = a
    \end{cases}
  \]
    整理可得$p+q = m+1-a$,因此$p,\ q$为以下一元二次方程的两个解
    \begin{equation*}
      x^2-(m+1-a)x+m = 0
    \end{equation*}
\end{solution}
\begin{exercise}
  \begin{enumerate}
    \item 证明$m>2,2|\varphi(m)$
    \begin{proof}
      \\方法一:\\
      $\varphi(m) = \prod(p_i^{\alpha_i}-p_i^{\alpha_i-1}) $\\
      分为两种情况考虑:\\
      1.m含有至少一个奇数因子\\
      2.m就是等于$2^i,i>1$\\
      方法二:\\
      m为奇数\\
      $1,2,3,\cdots, \frac{m-1}{2}, \frac{m+1}{2},\cdots,m$\\
      $(i,m)=1 \Rightarrow (m-i,m)=1$\\
    \end{proof}
    \item 证明,当$m|n$时,$\varphi(m)|\varphi(n)$。
    \begin{proof}
      \\$\varphi(m) = \prod \varphi(p_i^{\alpha_i})$\\
      $\varphi(n) = \prod \varphi(p_i^{\beta_i})$
    \end{proof}
  \end{enumerate}
\end{exercise}
\section{欧拉定理}
\subsection{欧拉定理和费马定理}
\begin{theorem}[欧拉定理]
  设m为正整数,$(a,m)=1$,那么$a^{\varphi(m)} \equiv 1(mod \ m)$。
\end{theorem}
\begin{proof}
  \\ 设$r_1,\ r_2,\ \cdots ,\ r_{\varphi(m)}$是模$m$的一个简化剩余系,因为$(a,m) = 1 $,由定理2.4可得,$ar_1,\ ar_2,\ \cdots,ar_{\varphi(m)}$也是模$m$的简化剩余系,因此对于任意$1 \leq i \leq \varphi(m)$,有且仅有唯一的$1 \leq j \leq \varphi(m)$使得$ar_j \equiv r_i(mod \ m)$,所以
  \begin{equation*}
    r_1r_2 \cdots r_{\varphi(m)} \equiv ar_1ar_2 \cdots ar_{\varphi(m)} \equiv a^{\varphi(m)}r_1r_2 \cdots r_{\varphi(m)}(mod \ m)
  \end{equation*}
  由定理2.4,$a^{\varphi(m)} \equiv 1 (mod \ m)$
\end{proof}
\begin{theorem}[费马小定理]
  设$p$为素数,那么对于任意整数$a$,$a^p \equiv a(mod \ p)$。
\end{theorem}
\begin{proof}
  \begin{enumerate}[(1)]
    \item 若$(a,p) =1 $,由欧拉定理,$a^{p-1} \equiv 1 (mod \ p)$。
    \item 若$(a,p) \neq 1$,那么$p | a$,所以$a \equiv 0(mod \ p)$,此时仍有$a^p \equiv a \equiv 0(mod \ p)$。
  \end{enumerate}
  欧拉定理和费马小定理反映了整数的幂关于模$m$的周期性。 \\
\end{proof}
\begin{exercise}
  假设今天是星期三,求$2^{20170226}$天后是星期几。
\end{exercise}
\begin{solution}
  \\ 因为$\varphi(7) = 6$,根据欧拉定理,$(2,7) =1,\ 2^6 \equiv 1 (mod \ 7)$,所以,对于任意整数$k \geq 0$,$2^{6k} \equiv 1 (mod \ 7)$。\\
  又因为$20170226 \equiv 2 (mod \ 6)$,此时$2^{20170226} = 2^{6k+2} = 2^{6k}2^2 \equiv 4(mod \ 7)$,故此$2^{20170226}$天后是星期天。
\end{solution}
\subsection{扩展欧拉定理}
\begin{theorem}[扩展的欧拉定理]
  设$m$为正整数,且$m= \prod_{i=0}^sp_i^{a_i}(a_i > 0)$,$a$为任意整数,在m的素因子中,仅当$1 \leq i \leq s^{\prime}$时,$p_i|a$,那么,对于任意$k \geq 0, \ t \geq \max_{1 \leq i \leq s^{\prime}}\{a_i\} $,均有$a^{k \varphi(m)+t} \equiv a^t(mod \ m)$。
\end{theorem}
\begin{conclusion}
  设$p_i$为素数,$a$为任意整数,那么,对于任意$k \leq 0, \ t \leq \beta_i$,均有$a^{k \varphi(\prod p_i^{\beta_i})+t} \equiv a^t(mod \ \prod p_i^{\beta_i})$。
\end{conclusion}
\vskip 0.5cm
\begin{exercise}
  $35^{20191443}(mod \ 75)$
\end{exercise}
\begin{solution}
  \\$35^{k\varphi(75)+t} \equiv 35^t(mod \ 3 \times 5^2)$
  \\因为$\varphi(75) = \varphi(3) \varphi(25) = 2 \times 5 \times 4 = 40$
  \\$20191443 \equiv 3 (mod \ 40)$
  \\所以可以得到$t = 3$,检验可知$3 >2,\ 3>1$
  \\ $(5^2 \times 7)^{k \varphi(m)+1} \equiv (5^2 \times 7)(mod \ 3 \times 5^2)$
\end{solution}
\vskip 0.5cm
\subsection{模重复平方算法}
 设$e = (e_{n-1}e_{n-2} \cdots e_1e_0)_2$是e的二进制表示,其中$e_i(0 \leq i \leq n-1)$为0或1,那么:
  \begin{align*}
   a^e &= a^{e_{n-1}2^{n-1}+e_{n-2}2^{n-2}+ \cdots + e_12+e_0} 
   \\ &= (a^{2^{n-1}})^{e_{n-1}} (a^{2^{n-2}})^{e_{n-2}} \cdots (a^2)^{e_1}a^{e_0}
  \end{align*}
 若我们以此计算
 \begin{align*}
  b_0 &\equiv a \equiv a^{2^0}(mod \ m)
  \\b_1 &\equiv {b_0}^2 \equiv {a^2}^1 (mod \ m)
  \\b_2 &\equiv {b_1}^2 \equiv {a^2}^2 (mod \ m)
  \\ \cdots
  \\b_{n-1} &\equiv {b_{n-2}}^2 \equiv {a^2}^{n-1} (mod \ m)
 \end{align*}
则$a^e \equiv \prod_{e_i =1}b_i(mod\ m)$。
\vskip 0.5cm
\begin{example}
  试求$2^{20170226}(mod\ 84375)$
\end{example}
\begin{solution}
  \\ 首先我们可以用欧拉定理化简,因为$84375 = 3^35^5$,所以,
  \begin{equation*}
    \varphi(84375) = 3^3 5^5 \times (1- \frac{1}{3})(1- \frac{1}{5}) = 4500
  \end{equation*}
  又因为$2^{20170226} \equiv 2^{20170226(mod\ 45000)} \equiv 2^{10226}(mod\ 84375)$,$10226 = (10011111110010)_2$,
  \\计算$b_0 \equiv 2,b_1 \equiv 4, b_2 \equiv 16 , \cdots , b_n = {b_{n-1}}^2, \cdots ,b_13 = 74146(mod 84375)$
  \\在计算过程中可以先和84375模的,就尽可能模,所以$b_i <84375$。
  \begin{align*}
    2^{10226} & \equiv b_1b_2 \cdots b_{13}
    \\ & \equiv 9019b_5b_6b_7b_8b_9b_{10}b_{13}
    \\ & \equiv 76999b_6 \cdots
    \\ & \equiv 17884b_7 \cdots
    \\ & \equiv 10354b_8 \cdots
    \\ & \cdots
    \\ & \equiv 65614(mod\ 84375)
  \end{align*}
\end{solution}
\vskip 0.5cm
\begin{note}
  $a^p \equiv a^{p(mod\ \varphi(m))}(mod\ m)$
\end{note}
\section{一次同余式}
\subsection{同余式及其解数}
\begin{definition}[同余式及其解数]
设$f(x) = \sum_{i=0}^n a_ix^i = a_nx^n + a_{n-1}x^{n-1}+ \cdots +a_1x +a_0$是一个整系数多项式,𝑚为正整数,那么称$f(x) \equiv 0(mod m)$为模m同余式。
如果$a_n \not\equiv (mod\ m)$,则称该同余式的次数为n。如果整数a满足$f(a) \equiv 0(mod \ m)$,称a为同余式的解。
\\ 由定理2.4之(5),若$b \equiv a(mod\ m)$,则$f(x) \equiv f(a) \equiv 0(mod\ m)$,所以b也是同余式的解,一般地,若a是同余式的解,那么模m剩余类$\bar{a}$中
的所有整数都是同余式的解,记作$x \equiv a(mod\ m)$。模m的完全剩余系
中同余式的解的个数称为同余式的解数。
\end{definition}
\vskip 0.5cm
例如,3满足同余式
$x^2 + 1 \equiv 0(mod\ 1 0)$,所以剩余类$x \equiv 3(mod\ 1 0)$中
的整数都是同余式的解,同样$x \equiv 7(mod\ 1 0)$也是同余式的解,该同余
式的解数为2。
\subsection{一次同余式}
\begin{theorem}
  设m为正整数,同余式$ax \equiv b(mod \ m)$有解的充要条件是$(a,m) | b$。在有解的情况下,解数为$(a,m)$,且若$x = x_0$是同余式的一个特解,那么同余式的所有解可以表示为$x \equiv x_0+ \frac{m}{(a,m)}t(mod \ m), \ t =0,1,2, \cdots, (a,m)-1$。
\end{theorem}
\begin{proof}
  \\$ax \equiv b(mod\ m)$有解也就是存在整数$y$使得$ax-b = my$有解的充要条件是$(a,m)|b$,所以同余式$ax \equiv b(mod\ m)$有解地充要条件是$(a,m)|b$。
  \\由定理1.8,若$x = x_0,\ y = y_0$是$ax-b = my $的一个解,那么$ax - b = my$的所有解可以表示为,
  \[
    \begin{cases}
      x = x_0+ \frac{m}{(a,m)}t
      \\y = y_0+ \frac{a}{(a,m)}t
      \\t \in Z
    \end{cases}
  \]
  我们将$x = x_0 + \frac{m}{(a,m)}t$写成$(a,m)$个模m的同余类为:
  \\$\overline{x_0},\ \overline{x_0+\frac{m}{(a,m)}},\ \overline{x_0+2*\frac{m}{(a,m)}} ,\ \cdots ,\ \overline{x_0+((a,m)-1)*\frac{m}{(a,m)}}$
  \\或者说原同余式的解为$x \equiv x_0 + \frac{m}{(a,m)} t(mod\ m),\ t = 0,\ 1,\ 2,\ \cdots ,\ (a,m)-1$,解数为$(a,m)$。
\end{proof}

\begin{example}
  试求同余式$6x \equiv 28(mod \ 32)$的解。
\end{example}
\begin{solution}
  \\因为$(6,32)=2 |28$,所以同余式有两个解。
  \begin{itemize}
    \item 原同余式可以等价于$3x \equiv 14(mod \ 16)$。
    \item 先求同余式$3x \equiv 1(mod \ 16)$的解,$x \equiv -5(mod \ 16)$。
    \item 由此$3x \equiv 14(mod \ 16)$的解为$x \equiv -5 \times 14 \equiv 10(mod\ 16)$,$x =10$是原同余式的一个特解。
    \item 原同余式的所有解为$x \equiv 10+16t(mod\ 16),\ t =0,\ 1$。
  \end{itemize}
\end{solution}
\subsection{逆元}
\begin{definition}
  设m为正整数,$(a,m)=1$,同余式$ax \equiv 1(mod\ m)$的解称为a模m的逆元,记作$x \equiv a^{-1}$,当$n \geq 0$时,我们用$a^{-n}(mod\ m)$来表示$a^n(mod\ m)$的逆元
\end{definition}
\begin{example}
  试求整数17模13的逆元。
\end{example}
\begin{solution}
  \\17模13的逆元即求同余式$17x \equiv 1 (mod\ 13)$的解。同余式可以等价地变形为$4x \equiv 1(mod\ 13)$,因为$4 \times 10 - 13 \times 3 =1$,所以$4x \equiv 1(mod \ 13)$的解为$x \equiv 10(mod\ 13)$,该剩余类的所有整数均为17模13的逆元。
\end{solution}
\begin{theorem}
  设m为正整数,$(a,m)=1$,那么$a^{\varphi(m)-1}$是a模m的逆元。
\end{theorem}
\begin{proof}
  \\ 因为$(a,m)=1$,根据欧拉定理,可以得到,
  \begin{align*}
    a^{\varphi(m)} \equiv 1(mod\ m)
  \end{align*}
  即$a \cdot a^{\varphi(m)-1} \equiv 1(mod\ m)$
  \\所以$x \equiv a^{\varphi(m)-1}$是同余式$ax \equiv 1 (mod\ m)$的解,由定义可知,$a^{\varphi(m)-1}$是a模m的逆元。
\end{proof}
\vskip 0.5cm
如果已知a模m的逆元$x \equiv a^{-1}(mod\ m)$,那么同余方程$ax \equiv b(mod\ m)$的解可以写成$x \equiv a^{-1}b(mod\ m)$。
\vskip 0.5cm
\begin{example}
  试证明:$(a^n)^{-1} \equiv (a^{-1})^n(mod\ m)$。
\end{example}
\begin{proof}
  \begin{itemize}
    \item 当$n \geq 0,\ a^n(a^{-1})^n \equiv 1(mod\ m)$,根据定义即可得证。
    \item 当$n < 0$,$(a^{-1})^n \equiv ((a^{-1})^{-n})^{-1} \equiv ((a^{-n})^{-1})^{-1} \equiv (a^n)^{-1}(mod\ m)$
  \end{itemize}
\end{proof}
\begin{theorem}[Wilson定理]
  设$p$是素数,那么$(p-1)! \equiv -1(mod\ p)$。
\end{theorem}
\begin{proof}
  \\ 证明逆定理
  \\$((p-1)!,p) = (-1,p )= 1$
  \\证明正定理
  \\若$p =2$,结论显然成立。
  \\设$p>2$,对于$1 \leq a \leq p-1$,因为$(a,p)=1$,所以$a$存在逆元$a^{\prime}$,由$ax \equiv 1 (mod\ m )$的解数为1,满足$1 \leq a^{\prime} \leq p-1$的逆元是唯一的。在$1,\ 2,\ \cdots,\ p-1$中,如果$a \neq a^{\prime}$,我们将$a$和$a^{\prime}$配对,得到$aa^{\prime} \equiv 1(mod\ m)$。如果$a = a^{\prime}$,得到$a^2 \equiv aa^{\prime} \equiv 1(mod\ p)$,只有$a =1$和$a =p-1$,所以,$(p-1)! \equiv 1 \times (p-1) \equiv -1(mod\ p)$。
\end{proof}
\vskip 0.5cm
\begin{conclusion}
  设p是奇素数,那么$(\frac{p-1}{2}!)^2 \equiv (-1)^{\frac{p+1}{2}}(mod\ p)$。
\end{conclusion}  
\
\begin{proof}
  \\根据Wilson定理
  \begin{align*}
    1 \times 2 \times \cdots \times \frac{p-1}{2} \times \frac{p+1}{2} \times \cdots \times (p-1) \equiv -1(mod\ p)
    \\ (\frac{p-1}{2}!)^2(-1)^{\frac{p-1}{2}} \equiv -1(mod\ p)
    \\ (\frac{p-1}{2}!)^2 \equiv (-1)^{\frac{p+1}{2}}(mod\ p)
  \end{align*}
\end{proof}
\begin{note}
  $a \equiv b (mod\ m) \Rightarrow (a,m)=(b,m)$
\end{note}
\section{中国剩余定理}
\subsection{中国剩余定理}
\begin{theorem}[CRT]
  设$m_1,\ m_2,\ ,\ \cdots ,\ m_s$为两两互素的正整数,$b_1,\ b_2,\ \cdots ,\ b_s$为任意整数,那么同余式组
  \[
    \left\{
      \begin{matrix}
        x \equiv b_1 (mod\ m_1)
        \\ x \equiv b_2(mod\ m_2)
        \\ \cdots 
        \\x \equiv b_s(mod\ m_s)
      \end{matrix}
    \right.
  \]
  模$M = m_1m_2 \cdots m_s$有唯一解$x \equiv \sum_{i=1}^sb_i \cdot \frac{M}{m_i}(\frac{M}{m_i})^{-1}(mod\ m_i)(mod\ M)$
\end{theorem}
\textbf{关于中国剩余定理的理解,可以参考刘铎老师的知乎:}
\href{https://zhuanlan.zhihu.com/p/44591114
}{https://zhuanlan.zhihu.com/p/44591114
}
\\
\begin{example}
  求解同余式
  \[
    \begin{cases}
      x \equiv 1(mod\ 2)
      \\ x \equiv 1(mod\ 3)
      \\ x \equiv 3(mod\ 5)
      \\ x \equiv 5(mod\ 7)
    \end{cases}
  \]
\end{example}
\vskip 0.5cm
\begin{solution}
  \\ $x \equiv 1 \times 105 \times 1 + 1 \times 70 \times 1 +3 \times 42 \times 3 +5 \times 30 \times 4 \equiv 1153 \equiv 103(mod\ 210)$
  \\如果将上述$105(mod\ 2)$的逆元1换成其他代表比如3,结果还正确,因为$105 \times (1+2t) = 105 \times 1 +Mt \equiv 105 (mod\ M)$
\end{solution}
\subsection{递推法}
\begin{example}
  求解同余式
  \[
    \begin{cases}
      x \equiv 1(mod\ 11)
     \\ x \equiv 2(mod\ 3)
     \\ x \equiv 3(mod\ 5)
    \\  x \equiv 4(mod\ 7)
    \end{cases}
  \]
\end{example}
\begin{solution}
  \\ $x \equiv 1(mod\ 11) \Rightarrow x-1 \equiv 0(mod\ 11)$
  \\ $x \equiv 2(mod\ 3)  \Rightarrow x -1 \equiv 1(mod\ 3)$
  \\ 结合上面两式: $x-1 \equiv 1 \times 11 \times (-1) \equiv -11(mod\ 33)$
  \\由此,$x+10 \equiv 0(mod\ 33)$
  \\再由$x \equiv 3 (mod\ 5) \Rightarrow x+10 \equiv 13 \equiv 3(mod\ 5)$
  \\结合上面两式:$x+10 \equiv 3 \times 33 \times 2 \equiv 198 \equiv 33(mod\ 165)$
  \\由此,$x-23 \equiv 0(mod\ 165)$
  \\再由$x \equiv 4(mod\ 7) \Rightarrow x-23 \equiv -19 \equiv 2(mod\ 7)$
  \\结合上面的两式:$x-23 \equiv 2 \times 165 \times 2 \equiv 660(mod\ 1155) \Rightarrow x \equiv 683(mod\ 1155)$

\end{solution}
\begin{conclusion}
  $\begin{cases}
    x \equiv 0(mod\ a)
    \\ x \equiv y(mod\ b)
  \end{cases}
  \Rightarrow
  x \equiv y \cdot a \cdot (a^{-1})(mod\ b) (mod\ ab)$
\end{conclusion}
\vskip 0.5cm
\begin{example}
  试用递推法,求韩信点兵的问题:
  \[\begin{cases}
    x \equiv 2(mod\ 3)
    \\ x \equiv 3(mod\ 5)
    \\ x \equiv 2(mod\ 2)
  \end{cases}\]
\end{example}
\subsection{应用}
\begin{example}
  计算$17^{20200301}(mod\ 105)$
\end{example}
\begin{solution}
  \\ $17^{20200301} \equiv (-1)^{20200301} \equiv -1 \equiv 2(mod\ 3)$
  \\$17^{20200301} \equiv 2^{20200301(mod\ 4)} \equiv 2(mod\ 5)$ 
  \\$17^{20200301} \equiv 3^{20200301(mod\ 6)} \equiv 3^{-1} \equiv 5(mod\ 7)$
  \\根据中国剩余定理,第1,2式合并:
  \\$17^{20200301} \equiv 2 \times 7 \times (-2) + 5 \times 15 \times 1 \equiv 75-28 \equiv 47 (mod\ 105)$
\end{solution}
\vskip 0.5cm
\begin{example}
  已知RSA解密密匙$(d,p,q)$,解密密文C。
\end{example}
\begin{solution}
  一般解法,$m \equiv C^d(mod\ pq)$
  中国剩余定理解法:
    \begin{equation*}
      m_1 \equiv C^{d(mod\ p-1)}(mod\ p)
    \end{equation*}
    \begin{equation*}
      m_2 \equiv C^{d(mod\ q-a)}(mod\ q)
    \end{equation*}
  \begin{equation*}
    m \equiv m_1pq^{-1}(mod\ q)+m_2pq^{-1}(mod\ p)(mod\ pq)
  \end{equation*}
\end{solution}
\section{同余式组解法}
\subsection{同余式组解数}
\begin{theorem}
  设$m_1,\ m_2,\ \cdots ,\ m_s$为两两互素的正整数,若对于$1 \leq i \leq s$,同余式$f_i(x) \equiv 0(mod\ m_i)$有$C_i$个解,那么,同余式组
  \[
    \left\{
    \begin{matrix}
      f_1(x) \equiv 0(mod\ m_1)
      \\f_2(x) \equiv 0(mod\ m_2)
      \\ \cdots
      \\f_s(x) \equiv (mod\ m_s)
    \end{matrix}
    \right.
  \]
  对于模$M = m_1m_2 \cdots m_s$有$C_1C_2 \cdots C_s$个解。
\end{theorem}
\begin{proof}
  \begin{itemize}
    \item 构造所有解
          \\设$f_i(x) \equiv 0(mod\ m_i)$的$C_i$个解为$x \equiv b_{i,1},\ b_{i,2},\ \cdots ,\ b_{i,C_i}(mod\ m_i)$,将这些解进行组合得到形式为
          \begin{equation*}
           x \equiv \sum_{i=1}^{s}b_i \cdot \frac{M}{m_i}(\frac{M}{m_i})^{-1}(mod\ m_i)(mod\ M)
          \end{equation*}
        的解,其中,$b_i$遍历$b_{i,1},\ b_{i,2},\ \cdots ,\ b_{i,C_i}$,解的个数为$C_1C_2 \cdots C_s$。
    \item 两两互不同余\\
    下面我们证明这些解关于模M是两两互不同余的。\\
    若$x^{\prime} \equiv  \sum_{i=1}^s b_i^{\prime} \cdot \frac{M}{m_i}(\frac{M}{m_i})^{-1}(mod\ m_i)(mod M)$是其中的一个解,且$x^{\prime} \equiv x(mod\ M)$,根据定理2.2,
    $
      \left\{
        \begin{matrix}
          x \equiv x^{\prime}(mod\ m_1)
          \\ x\equiv x^{\prime}(mod\ m_2)
          \\ \cdots
          \\x \equiv x^{\prime}(mod\ m_s)
        \end{matrix}
      \right.
    $
    ,可以得到
    $
     \left\{
       \begin{matrix}
         b_1 \equiv b_1^{\prime}(mod\ m_1)
         \\b_2 \equiv b_2^{\prime}(mod\ m_2)
         \\ \cdots
         \\b_s \equiv b_s^{\prime}(mod\ m_s)
       \end{matrix}
     \right.
    $
    说明,只要$b_i$中的任意一个发生变化,得到的解都是不同的。
  \end{itemize}
\end{proof}
\subsection{模同一素数幂}
\begin{theorem}
  设p为素数,$i_1 \geq i_2 \geq \cdots \geq i_s,\ b_1,\ b_2,\ \cdots ,\ b_s$为任意整数,那么同余式组
  \[
    \left\{
      \begin{matrix}
        x \equiv b_1(mod\ p^{i_1})
        \\ x \equiv b_2(mod\ p^{i_2})
        \\ \cdots
        \\ x \equiv b_s(mod\ p^{i_s})
      \end{matrix}
    \right.
  \]
  有解的充要条件是
  \[\left\{
    \begin{matrix}
      b_1 \equiv b_2(mod\ p^{i_2})
    \\b_1 \equiv b_3(mod\ p^{i_3})
    \\ \cdots
    \\ b_1 \equiv b_s(mod\ p^{i_s})
    \end{matrix}
  \right.\]
  如果有解,其解为$x \equiv b_1(mod\ p^{i_1})$。
\end{theorem}
\begin{example}
  试判断同余式组$\begin{cases}
    x \equiv 9(mod\ 15)
    \\ x \equiv 49(mod\ 50)
    \\x \equiv -41(mod\ 140)
  \end{cases}$
  是否有解,如果有解,求出其解。
\end{example}
\begin{solution}
  \\ 原同余式可以等价变形为$\begin{cases}
    x \equiv 9(mod\ 3)
    \\x \equiv 9(mod\ 5)
    \\x \equiv 49(mod\ 5^2)
    \\x \equiv 49(mod\ 2)
    \\x \equiv -41(mod\ 7)
    \\x \equiv -41(mod\ 2^2)
    \\x \equiv -41(mod\ 5)
  \end{cases}$
  \\ $49 \equiv 9 \equiv -41(mod\ 5),\ -41 \equiv 49(mod\ 2)$有解
  \\ 可以等价地变形为
  \\$\begin{cases}
    x \equiv 9(mod\ 3)
    \\ x \equiv 49(mod\ 5^2)
    \\ x \equiv -41(mod\ 7)
    \\ x \equiv -41(mod\ 2^2)
  \end{cases}$
  ,进一步化简为$\begin{cases}
    x \equiv 0(mod\ 3)
    \\x \equiv -1(mod\ 5^2)
    \\x \equiv 1(mod\ 7)
    \\x \equiv 3(mod\ 2^2)
  \end{cases}$
  最后求得$x \equiv 99(mod\ 2100)$(用CRT)。
\end{solution}
\subsection{一般同余式求解}
对于合数$m = \prod_{i=1}^sp_i^{\alpha_i}$,求同余式$f(x) \equiv 0(mod\ m)$的解等价于求同余式组
\[\left\{
  \begin{matrix}
    f(x) \equiv 0(mod\ p_1^{\alpha_1})
    \\f(x) \equiv 0(mod\ p_2^{\alpha_2})
    \\ \cdots
    \\f(x) \equiv 0(mod\ p_s^{\alpha_s})
  \end{matrix}
\right.\]
的解。
\vskip 0.5cm
\begin{exercise}
  \begin{itemize}
    \item 试证明同余式组$\begin{cases}
    x \equiv a(mod\ m)
    \\x \equiv b(mod\ n)
    \end{cases}$
    有解的充要条件是$(m,n) \mid a-b$
    \item 求解同余式组$\begin{cases}
      x^2 \equiv 1(mod\ 7)
      \\4x \equiv 4(mod\ 6)
      \\x \equiv 4(mod\ 9)
    \end{cases}$
  \end{itemize}
\end{exercise}
\section{模为素数幂的高层同余式}
\begin{definition}
  设$f(x) = a_nx^n+a_{n-1}x^{n-1}+ \cdots +a_1x+a_0$是一个正系数多项式,其一阶导式$f^{\prime}(x) = na_mx^{n-1}+(n-1)a_{n-1}x^{n-2}+ \cdots +a_1$。$f(x)$的一阶导式也可以记为$f^{(1)}(x)$,依次地,定义其m阶导式为$f^{(m)}(x) = (f^{(m-1)}(x))^{\prime}$。
\end{definition}
\begin{theorem}[构造所有解]
  设p为素数,$k \geq 1$,那么$x \equiv x_k(mod\ p^k)$是同余式
  $f(x) \equiv (mod\ p^k)$的一个解,那么在这个剩余类中,
  \begin{itemize}
    \item 若$(p,f^{\prime}(x_k)) = 1$,同余式$f(x) \equiv 0(mod\ p^{k+1})$有唯一解。
    \item 若$p \mid f^{\prime}(x_k)$,当$f(x_k) \not\equiv 0(mod\ p^{k+1})$时,同余式$f(x) \equiv 0(mod\ p^{k+1})$无解;当$f(x_k) \equiv 0(mod\ p^{k+1})$时,同余式$f(x) \equiv 0(mod\ p^{k+1})$有p个解
  \end{itemize}
\end{theorem}
\begin{proof}
  \begin{itemize}
    \item 根据定理2.2,同余式$f(x) \equiv 0(mod\ p^{k+1})$的解一定满足$f(x) \equiv 0(mod\ p^k)$所以$f(x) \equiv 0(mod\ p^{k+1})$的解可以从$f(x) \equiv 0(mod\ p^k)$𝑓(𝑥) ≡ 0的解中进行筛选而得。
    \item 因为$x \equiv x_k(mod\ p^k)$是同余式$f(x) \equiv 0(mod\ p^k)$的一个解,我们将从
    \begin{equation*}
      x = x_k+p^kt, t \in Z
    \end{equation*}
    这个剩余类中筛选出$f(x) \equiv 0(mod\ p^{k+1})$的解。
  \end{itemize}
  将$x = x_k+p^kt$代入$f(x) \equiv 0(mod\ p^{k+1})$,有$f(x_k+p^kt) \equiv 0(mod\ p^{k+1})$,用泰勒公式展开得到
  \begin{equation*}
    f(x_k)+f^{\prime}(x_k)p^kt+ \sum_{i =2}^{n} \frac{f^{(i)}(x_k)p^{ik}}{i!} t^i \equiv 0(mod\ p^{k+1})
  \end{equation*}
  进一步化简可得
  \begin{equation*}
    f(x_k)+f^{\prime}(x_k)p^kt \equiv 0(mod\ p^{k+1})
  \end{equation*}
  由$f(x_k) \equiv 0(mod\ p^k)$,有$p^k \mid f(x_k)$,所以上式可以进一步化简为
  \begin{equation*}
    f^{\prime}(x_k)t \equiv - \frac{f(x_k)}{p^k}(mod\ p)
  \end{equation*}
  \begin{enumerate}[(1)]
    \item 若$(f^{\prime}(x_k),\ p) =1$,上式有唯一解$t \equiv - \frac{f(x_k)}{p^k}(f^{\prime}(x_k))^{-1}(mod\ p)$,从而筛得$f(x) \equiv 0 (mod\ p^{k+1})$的解为
    \begin{align*}
      x & \equiv x_0 - \frac{f(x_k)}{p^k}((f^{\prime}(x_k))^{-1}(mod\ p))p^k
      \\& \equiv x_k - f(x_k)((f^{\prime})^{-1}(mod\ p))(mod\ p^{k+1})
    \end{align*}
    \item  若$p \mid f^{\prime}(x_k)$,当$f(x_k) \not\equiv 0(mod\ p^{k+1})$时,上式无解;
    \\当$f(x_k) \equiv 0(mod\ p^{k+1})$时,上式有$p$个解,也就是说$x = x_k+p^kt,\ t \in Z,\ x \equiv x_k+p^kt(mod\ p^{k+1}), t =0,\ 1,\ \cdots ,\ p-1$。
  \end{enumerate}
\end{proof}
\vskip 0.5cm
\begin{conclusion}
  设p为素数,若$x \equiv x_1(mod\ p)$是同余式$f(x) \equiv 0(mod\ p)$的一个解,且满足$(f^{\prime}(x_1),\ p) = 1$,那么对于任意正整数$k>1,\ f(x) \equiv 0(mod\ p^k)$的满足$x \equiv x_1(mod\ p)$的解$x_k$可以通过如下递推公式得到:
  \begin{equation*}
    x_i \equiv x_{i-1}-f(x_{i-1})((f^{\prime})^{-1}(mod\ p))\ (mod\ p^i),\quad i = 2,\ 3,\ \cdots ,\ k
  \end{equation*}
\end{conclusion}
\begin{proof}
  \\利用数学归纳法可以证明。
\end{proof}
\section{高次同余式求解方法}
\subsection{唯一解}
\begin{example}
  试求同余式$x^3+4x^2+1 \equiv 0(mod\ 3^5)$的解
\end{example}
\begin{solution}
  \\令$f(x) = x^3+4x^2+1$,则$f^{\prime}(x) = 3x^2+8x \equiv 2x(mod\ 3)$
  \\由同余式$f(x) \equiv 3(mod\ 3)$得到解$x_1 \equiv (mod\ 3)$
  \\所以,$(f^{\prime}(x_1))^{-1} \equiv 2^{-1} \equiv 2(mod\ 3)$
  \\根据定理的推论有
  \begin{align*}
    x_2 &\equiv x_1-2f(x_1) \equiv 7(mod\ 3^2)
    \\x_3 &\equiv x_2-2f(x_2) \equiv 7(mod\ 3^3)
    \\x_4 &\equiv x_3-2f(x_3) \equiv 61(mod\ 3^4)
    \\x_5 &\equiv x_4-2f(x_4) \equiv 142(mod\ 3^5)
  \end{align*}
  所以,原同余式的解为$x \equiv 142(mod\ 3^5)$
\end{solution}
\subsection{无解}
\begin{example}
  试求同余式$x^2+p \equiv 0(mod\ p^3)$,其中$p$为奇素数。
\end{example}
\begin{solution}
  \\令$f(x) = x^2+p$,则$f^{\prime}(x) \equiv 2x(mod\ p)$
  \\ 考虑$f(x) \equiv 0(mod\ p)$,它的唯一解为$x_1 \equiv 0(mod\ p)$
  \\$f^{\prime}(x_1) \equiv 0(mod\ p)$,所以$p \mid f^{\prime}(x_1)$
  \\因为$f(x_1) = p$,所以$p^2 \not\mid f(x_1)$,根据定理,同余式$x^2+p \equiv 0(mod\ p^2)$无解,原同余式也一定无解。
\end{solution}
\subsection{多个解}
\begin{example}
  试求解同余式$x^2-p^2 \equiv 0(mod\ p^3)$,其中$p$为奇素数。
\end{example}
\begin{solution}
  \\令$f(x)=x^2-p^2$,则$f^{\prime}(x) \equiv 2x(mod\ p)$
  \\考虑$f(x) \equiv 0(mod\ p)$,它有唯一解$x_1 \equiv 0(mod\ p)$
  \\$f^{\prime}(x_1)(mod\ p)$,所以$p \mid f^{\prime}(x_1)$
  \\$f(x_1) = -p^2$,所以$p^2 \mid f(x_1)$,根据定理,同余式$x^2-p^2 \equiv 0(mod\ p^2)$有p个解,$x_2 = 0,\ p,\ 2p,\ \cdots ,\ (p-1)p(mod\ p^2)$
  \\$f(x_2) = (ip)^2 - p^2 = (i^2-1)p^2$,仅当$i=1,\ -1$时,$p^3 \mid f(x_2)$,所以,同余式$x^2-p^2 \equiv 0(mod\ p^3)$的解为:$x_3 \equiv \pm p,\ p^2 \pm p,\ 2p^2 \pm p,\ \cdots \ (p-1)p^2 \pm p(mod\ p^3)$
\end{solution}
\subsection{复杂情况}
\begin{itemize}
  \item 转化为素数幂(简化模)
  \\对于合数$m = \prod_{i=1}^s p_i^{\alpha_i}$,求同余式$f(x) \equiv 0(mod\ m)$的解等价于求同余式组
  \[
      \left\{
        \begin{matrix}
          f(x) \equiv 0(mod\ p_1^{\alpha_1})
          \\f(x) \equiv 0(mod\ p_2^{\alpha_2})
          \\ \cdots
          \\f(x) \equiv 0(mod\ p_s^{\alpha_s})
        \end{matrix}
      \right. 
  \]
  的解
  \vskip 0.5cm
  \item 扩展欧拉定理之结论(降次)
  \\设p为素数,a为任意整数,那么,对于任意$k \geq 0,\ t \geq \beta$,均有$a^{k\varphi(p^{\beta})+t} \equiv a^t(mod\ p^{\beta})$。
\end{itemize}
\vskip 0.5cm
\begin{example}
  求解同余式$x^254+2x^3 \equiv 0(mod\ 135)$
\end{example}
\begin{solution}
  \\ \[
    \begin{cases}
      x^2+2x^3 \equiv 0(mod\ 5) \Leftrightarrow x^2(1+2x) \equiv 0(mod\ 5) \Rightarrow x \equiv 0,\ 2 (mod\ 5)
      \\x^{254}+2x^3 \equiv 0(mod\ 27) \Leftrightarrow x^{20}+2x^3 \equiv 0(mod\ 27)
    \end{cases}
  \]
  令$f(x) = x^{20}+2x^3$,则$f(x) \equiv 0(mod\ 3)$得解$x_1 \equiv 0,1(mod\ 3)$
  \\$f^{\prime}(x) \equiv 2x(mod\ 3),\ f^{\prime}(1) \equiv 2(mod\ 3),\ {f^{\prime}(1)}^{-1} \equiv 2(mod\ 3)$,由$x_1 \equiv 1(mod\ 3)$,所以
  \\ $x_2 \equiv 1-2*f(1)(mod\ 9) \equiv 4(mod\ 9)$
  \\ $x_3 \equiv 4-3*f(4)(mod\ 27) \equiv 4-2(4^2+2*4^3) \equiv 13(mod\ 27)$
  \\ 由$x_1 \equiv 0(mod\ 3)$,由于$3t +x_1$均满足$x^{254}+2x^3 \equiv 0(mod\ 27)$,所以$x_3 \equiv 0,3,6,9, \cdots , 24(mod\ 27)$
  \\综上得到
  \[
    \begin{cases}
      x \equiv a \equiv 0(mod\ 5)
      \\ x \equiv b \equiv 0,3,6,9,12,13,15,18,21,24(mod\ 27)
    \end{cases}\]
    综上所述,根据中国剩余定理,所有解为:
    \\$x \equiv a \cdot 27 \cdot ((27)^{-1}(mod\ 5)) + b \cdot 5 \cdot 11(mod\ 135)$
    \\$x \equiv 0,30,60,90,120,40,15,45,75,\cdots$
\end{solution}
\begin{problemset}
  \item 试证明:如果$r_1,\ r_2,\ \cdots ,\ r_{\varphi(m)}$是模$m(m>2)$的一个简化剩余系,那么$\sum_{i=1}^{\varphi(m)}r_i \equiv 0(mod\ m)$。
  %\begin{proof}
   % \\ 对于任意$r_i$,我们考虑剩余类$-r_i(mod\ m)$
    %\\ $(1)(-r_i,m) = (r_i,m) = 1$,简化剩余类
    %\\ $(2)r_i \not\equiv -r_i(mod\ m)$,否则,$2r_i \equiv 0(mod\ m)$
  %\end{proof}
  \item 试证明:如果n是正奇数,那么$\sum_{i=1}^{n-1}i^3 \equiv 0(mod\ n)$。
  \item 试判断128749832749837759345是否是11和13的倍数。
  \item 试计算$13^{20170226}(mod\ 72)$。
  \item 试证明:若$N = \prod_{i=1}^n p_i^{\alpha_i},\ (a,\ N) = 1$,那么
  \begin{equation*}
    a^{[\varphi(p_1^{\alpha_1}),\varphi(p_2^{\alpha_2}),\cdots,\varphi(p_n^{\alpha_n})]} \equiv 1(mod\ N)
  \end{equation*}
  \item 试证明:正整数$N>1$满足$\varphi(N) =2^n$的充要条件是N有分解式$N=2^m\prod_{i=1}^SF_i,\ m\geq 0,\ F_i$为互不相等的Fermat素数。
  \item RSA加密算法。设$N = pq,\ p,\ q $为不相等的素数,正整数e满足$(e,\varphi(N)) = 1$,d满足$ed \equiv 1(mod\ \varphi(N))$,试证明:对于任意明文$0 \leq m <N$,若加密算法为$c \equiv m^e(mod\ N)$,那么由密文c,可以通过计算$c^d(mod\ N)$解密得到明文m。
  \item 试证明:正整数$N>1$对于任意整数$0 \leq a <N$和$k>0$均有$a^{k\varphi(N)+1} \equiv a(mod\ N)$的充要条件是N有分解式$N = \prod_{i=1}^np_i$,$p_i$为互不相等的素数。
  \item 设模$m$的简化剩余系为$r_1,\ r_2,\ \cdots,\ r_{\varphi(m)}$,试证明:$(\prod_{i=1}^{\varphi(m)})^2 \equiv 1(mod\ m)$。
  \item 试求解一次同余式$256x \equiv 28(mod\ 400)$。
  \item 如果p是素数,且$p \equiv 1(mod\ 4)$,那么同余式$x^2 \equiv -1(mod\ p)$有两个不同余的解
  \begin{equation*}
    x \equiv \pm \frac{p-1}{2}!(mod\ p)
  \end{equation*}
  \item 试计算$3^{20100416}(mod\ 35),\ 3^{20150410}(mod\ 35 \times 27)$。
  \item 试证明:如果$(a,32760) = 1$,那么$a^{12} \equiv 1(mod\ 32760)$
  \item 求解同余式$x^{10}+x^2+x+1 \equiv 0(mod\ 2^5)$。
  \item 求解同余式$x^2+px \equiv 0(mod\ p^3)$,其中p为奇素数。
\end{problemset}
\chapter{域}
\section{域的定义与性质}
\subsection{域的定义}
\begin{definition}
  设$\mathbb{F}$是一个非空集合,对其上的定义了两种运算,分别叫做加法和乘法,记作$"+"," \cdot "$,对于$\mathbb{F}$中的任意两个元素$a,b,$均有$a+b \in \mathbb{F},\ a \cdot b \in \mathbb{F}$($\mathbb{F}$对于加法和乘法自封闭,$a+b,\ a \cdot b$分别称为两个元素的和与积,$a \cdot b$通常记作$ab$),我们称$\mathbb{F}$对于所规定的加法和乘法成为一个域,如果其元素满足以下运算规则:
  \begin{itemize}
    \item $\mathbb{F}$中所有元素对于加法形成一个加法交换群;
    \item $\mathbb{F}$中所有非零元素($\mathbb{F}^{*}$)对于乘法形成一个乘法交换群;
    \item 对于任意$a,\ b,\ c \in \mathbb{F},\ a(b+c) = ab + ac$(乘法对加法的分配律).
  \end{itemize}
\end{definition}
\vskip 0.5cm
\textbf{一个域至少有两个元素,即\textcolor{pink}{加法群的零元}和\textcolor{pink}{乘法群的单位元},他们分别
称为域的零元和单位元,记作0和1,单位元有时也记作𝑒。 当称一个集合
是域的时候,除了需要指明集合本身的元素外,还要指明集合上定义的加
法和乘法。}
\vskip 0.5cm
\textbf{当域的元素个数有限时称为\textcolor{pink}{有限域},或者\textcolor{pink}{伽罗华域},否则称为\textcolor{pink}{无限域}。
常见的有理数集合$\mathbb{Q}$,实数集合$\mathbb{R}$和复数集合$\mathbb{C}$按照其上定义的加法和乘
法都形成域,分别叫做有理数域,实数域和复数域。
}
\vskip 0.5cm
\begin{example}
  以下集合按照定义的加法和乘法均不形成域:
  \begin{enumerate}[(1)]
    \item 全体整数形成的集合$\mathbb{Z}$,加法和乘法分别为$\mathbb{Z}$上的加法和乘法;
    \item 集合$\{a+b \sqrt[3]{2} \mid a,\ b \in \mathbb{Q}\}$,加法和乘法分别为实数域$\mathbb{R}$上的加法和乘法;
    \\ \textcolor{pink}{但是 $a+b \sqrt[3]{2} + c \sqrt[3]{4}$是一个域;}
    \item $\mathbb{Z}_{m} = \{0,\ 1,\ \cdots ,\ m-1 \}$,m为合数,加法和乘法分别为模m的加法和乘法。
  \end{enumerate}
\end{example}
\begin{theorem}
  设$\mathbb{F}$是一个域,那么
  \begin{enumerate}[(1)]
    \item 对于任意$a \in \mathbb{F},\ 0a = a0 = 0$。
    \item 任意$a,\ b \in \mathbb{F},\ if \ ab =0,\ \Rightarrow a = 0 \ or \ b = 0$。
  \end{enumerate}
\end{theorem}
\subsection{子域和扩域}
\begin{definition}
  设$\mathbb{F}$是一个域,$\mathbb{F}_0$是$\mathbb{F}$的非空子集,如果对于$\mathbb{F}$上的加法和乘法,$\mathbb{F}_0$也是一个域,则称$\mathbb{F}_0$是$\mathbb{F}$的子域,$\mathbb{F}$是$\mathbb{F}_0$的扩域,记作$\mathbb{F}_0 \subset \mathbb{F}$。
\end{definition}
\textbf{例如:$\mathbb{Q} \subset \mathbb{Q}[\sqrt{2}] \subset \mathbb{R} \subset \mathbb{R}[\sqrt{-2}] \subset \mathbb{C}$。}
\subsection*{判断方法}
\begin{theorem}
  设$\mathbb{F}_0,\ \mathbb{F}_0^{*}$均是域$\mathbb{F}$的非空子集,当且仅当以下条件成立时$\mathbb{F}_0$是域$\mathbb{F}$的子域:
  \begin{enumerate}[(1)]
    \item 任意$a,\ b \in \mathbb{F}_0$,都有$-a,\ a+b \in \mathbb{F}_0$;
    \item 任意非零元素$a,\ b \in \mathbb{F}_0$,都有$a^{-1},\ ab \in \mathbb{F}_0$。
  \end{enumerate}
\end{theorem}
\begin{proof}
  \begin{enumerate}[(1)]
    \item 说明$\mathbb{F}_0$是$\mathbb{F}$的加法子群
    \item 说明$\mathbb{F}_0^{*}$是$\mathbb{F}^{*}$的乘法子群。\\ 而乘法对加法的分配律在$\mathbb{F}$中成立,那么在$\mathbb{F}_0$中也必然成立,根据定义$\mathbb{F}_0$是一个域。
  \end{enumerate}
\end{proof}
\begin{example}
  \begin{enumerate}[(1)]
    \item $\mathbb{Q}[\sqrt{2}]=\{a+b \sqrt{2} \mid a,\ b \in \mathbb{Q} \}$,加法和乘法分别为实数域$\mathbb{R}$上的加法和乘法;
    \item $\mathbb{Z}_p = \{0,1, \cdots , p-1 \}$,p为素数,加法和乘法分别为模p的加法和乘法;
  \end{enumerate}
\end{example}
\section{域的特征}
\subsection{特征的定义}
\begin{definition}
  设$\mathbb{F}$是一个域,如果存在正整数m,使得对于任意$a \in \mathbb{F}$均有$ma = 0$,则在所有$m$中,最小的正整数称为\textcolor{blue}{域$\mathbb{F}$的特征};否则,如果不存在正整数m,使得对于任意$a \in \mathbb{F}$均有$ma =0$,则称域$\mathbb{F}$的\textcolor{blue}{特征为0}。域$\mathbb{F}$的特征记作$char(\mathbb{F})$。
\end{definition}
\subsection{域的同构}
\begin{definition}
  设$\mathbb{F},\ \mathbb{K}$是两个域,如果存在$\mathbb{F}$到$\mathbb{K}$上的一一映射$\delta$,使得对于任意$a,\ b \in \mathbb{F}$,均有
  \begin{equation*}
    \delta (a+b) = \delta (a)+\delta (b),\ \delta (ab) = \delta (a)\delta (b)
  \end{equation*} 
  则称$\delta $为$\mathbb{F}$到$\mathbb{K}$上的\textcolor{blue}{同构映射},此时称域$\mathbb{F},\ \mathbb{K}$同构,记作$\mathbb{F} \cong \mathbb{K}$。如果$\mathbb{F}=\mathbb{K}$,则称$\delta $为\textcolor{blue}{自同构映射},特别地,若进一步对于任意$a \in \mathbb{F}$均有$\delta (a) = a$,则称$\delta $为\textcolor{blue}{恒等自同构映射}。
\end{definition}
\subsection{素域} 
\textbf{一个域的最小子域称为该域的素域}
\begin{theorem}
  设$\mathbb{F}$是一个域,如果$char(\mathbb{F})$为正整数,则必为某个素数$p$。特征为素数$p$的域的素域与$\mathbb{Z}_p$同构,特征为0的域的素域与$\mathbb{Q}$同构。
\end{theorem}
\begin{proof}
  反证法。假设$char(\mathbb{F}) = m >0$,m为合数。设p是m的最小素因子,$m = ps,\ 1 < p<m,\ 1<s<m$,则$(ps)e = me = 0$,而$(ps)e = (pe)(se)$,根据定理3.1,$pe = 0$或者$se = 0$。但是对于任意$a \in \mathbb{F},\ pa = (pe)a,\ sa = (se)a$,所以必有$pa = 0$或者$sa = 0$,这与$m$的最小性相矛盾。
  \\
  当$char(\mathbb{F})=p$时,可以验证$\mathbb{F}_0=\{0,\ e,\ 2e,\ ,\cdots ,\ (p-1)e \}$是域$\mathbb{F}$的最小子域,而映射$\delta: \mathbb{F}_0 \rightarrow \mathbb{Z}_p,\ \delta(ie) = i$为域同构映射。
  \\
  当$char(\mathbb{F}) = 0$时,可以验证$\mathbb{F}_0 = \{(ae)(be)^{-1} \mid a,b \in Z,b \neq 0\}$是域$\mathbb{F}$的最小子域,而映射$\delta: \mathbb{F}_0 \rightarrow \mathbb{Q},\ \delta((ae)(be)^{-1})=\frac{a}{b} \in \mathbb{Z},\ b \neq 0$为域同构映射。
\end{proof}
\begin{exercise}
  试证明对于任何非负整数n,在特征为p的有限域$\mathbb{F}$上定义的映射$\delta_n:\mathbb{F} \rightarrow \mathbb{F},\ \delta_n(\alpha)=\alpha^{p^n}$均是$\mathbb{F}$的自同构。
\end{exercise}
\section{二项式定理}
\subsection{二项式定理}
\begin{theorem}[二项式定理]
  设$\mathbb{F}$是一个域,$a,\ b \in \mathbb{F}$,则对于任意正整数n,$(a+b)^n = \sum_{i=0}^{n} \binom{n}{i}a^{n-i}b^i$
\end{theorem}
\subsection{特征幂的二项式定理}
\begin{theorem}
  设$\mathbb{F}$是一个域,$char(\mathbb{F})=p$,则对于任意$a,\ b \in \mathbb{F},\ n \geq 0$,均有
  \begin{equation*}
    (a \pm b)^{p^n} = a^{p^n} \pm b^{p^n}
  \end{equation*}
\end{theorem}
\begin{proof}
  $n = 0$时结论成立。\\
  下面对$n>0$使用数学归纳法证明$(a+b)^{p^n} = a^{p^n}+b^{p^n}$。\\
  $n =1$ 时,因为$p \mid \binom{p}{i}(0<i<p)$,根据二项式定理,$(a+b)^p = a^p +b^p$。\\
  假设$n = k$时结论成立,当$n = k+1$时,
  \begin{align*}
    (a+b)^{p^{k+1}}=((a+b)^{p^k})^p = (a^{p^k}+b^{p^k})^p = a^{p^{k+1}}+b^{p^{k+1}},
    \\ (a-b)^{p^n} = (a+(-b))^{p^n} = a^{p^n}+(-1)^{p^n}b^{p^n},
  \end{align*}
  当$p \neq 2$时,$(-1)^{p^n} = -1$,所以,$(a-b)^{p^n} = a^{p^n} - b^{p^n}$;
  \\当$p = 2$时,$(-1)^{p^n} = 1 = -1$,仍有$(a-b)^{p^n} = a^{p^n}-b^{p^n}$。
\end{proof}
\subsection{域上的多项式}
\begin{definition}
  对于非负整数$i,\ a_ix^i,\ a_i \in \mathbb{F}$表示域$\mathbb{F}$上文字为$x$的\textcolor{blue}{单项式},我门称形式和$f(x)=a_nx^n+a_{n-1}x^{n-1}+ \cdots + a_1x^1 + a_0x^0,\ a_i \in \mathbb{F}$为域$\mathbb{F}$上的文字为$x$的多项式,简称为域$\mathbb{F}$上的多项式。
\end{definition}
\subsection{多项式的加法和乘法}
在$\mathbb{F}$上可以定义加法"+"和乘法"$\cdot$",
设$f(x)= \sum_{i=0}^na_ix^i,\ g(x)= \sum_{i=0}^mb_ix^i,n \geq m,\ b_{m+1}= b_{m+2}=\cdots =b_n=0$,则可定义
\begin{align*}
  f(x)+g(x)=\sum_{i=0}^n(a_i+b_i)x_i,
  \\
  f(x) \cdot g(x)= \sum_{j=0}^{m+n}(\sum_{i=0}^ja_ib_{j-i})x^j
\end{align*}
关于多项式的次数,下面结论成立:
\begin{align*}
  deg(f(x)+g(x)) \leq max\{degf(x),\ degg(x)\}\\
  deg(f(x)g(x))=degf(x)+degg(x)
\end{align*}
$\mathbb{F}[x]$按照上面的定义的加法和乘法\textbf{不是域},因为除了$\mathbb{F}$中的非零元素,$\mathbb{F}[x]$中的其他元素均没有逆元。
\vskip 0.5cm
$f(x),\ g(x)$是特征为p的域上的多项式,那么
\begin{equation*}
  (f(x) \pm g(x))^{p^n}=f(x)^{p^n} \pm g(x)^{p^n}
\end{equation*}
\vskip 0.5cm
\textbf{$\mathbb{F}(x)=\{\frac{f(x)}{g(x)} \mid f(x),g(x) \in \mathbb{F}[x],\ g(x) \neq 0 \}$
按照以下运算规则形成一个域,其中$\frac{f(x)}{g(x)}+\frac{s(x)}{t(x)}=\frac{f(x)t(x)+s(x)g(x)}{g(x)t(x)},\\ \frac{f(x)}{g(x)} \frac{s(x)}{t(x)}=\frac{f(x)s(x)}{g(x)t(x)}$,
且$\frac{f(x)}{g(x)}=\frac{f_1(x)}{g_1(x)}$当且仅当$f(x)g_1(x)= g(x)f_1(x)$。特别地,当$\mathbb{F}$是特征为p的有限域时,$\mathbb{F}(x)$是特征为p的无限域。}
\section{多项式的辗转相除法}
\subsection{带余除法算式}
\begin{theorem}
  设$f(x),g(x)$为域$\mathbb{F}$上的两个多项式,$g(x) \neq 0$,则存在唯一的一对多项式$q(x), \ r(x)$,使得:
  \begin{equation*}
    f(x)=q(x)g(x)+r(x),\quad deg\ r(x) < deg\ g(x)
  \end{equation*}
  $r(x)$称为$f(x)$被$g(x)$除所得的\textbf{余式},记作$(f(x))_{g(x)} = r(x)$。
\end{theorem}
\begin{proof}
  设$f(x)=a_nx^n+\cdots + a_1x +a_0,\ g(x)=b_mx^m+\cdots + b_1x+b_0$。
  \begin{itemize}
    \item \textcolor{pink}{归纳基础}\quad 当$n<m$时,取$q(x)=0,\ r(x)=f(x)$,结论成立。
    \item \textcolor{pink}{归纳假设}\quad 假设当$n<k(k \geq m)$时结论均成立。
  \end{itemize}
\end{proof}
\begin{theorem}
  设$f_1(x),\ f_2(x),\ g(x)$为域$\mathbb{F}$上的多项式,$g(x) \neq 0$,则
  \begin{align*}
    (f_1(x)+f_2(x))_{g(x)} = (f_1(x))_{g(x)}+(f_2(x))_{g(x)}\\
    (f_1(x)f_2(x))_{g(x)} = ((f_1(x))_{g(x)}(f_2(x))_{g(x)})_{g(x)}
  \end{align*}
\end{theorem}
\begin{proof}
  \begin{align*}
    (f_1(x)+f_2(x))_{g(x)}&=(q_1(x)g(x)+(f_1(x))_{g(x)}+q_2(x)g(x)+(f_2(x)_g(x)))_{g(x)}\\
    &=((q_1(x)+q_2(x))g(x)+(f_1(x))_{g(x)}+(f_2(x))_{g(x)})_{g(x)}\\
    &=(f_1(x))_{g(x)}+(f_2(x))_{g(x)}
    \\ \\
    (f_1(x)f_2(x))_{g(x)}&=((q_1(x)q_2(x)g(x)+q_1(x)(f_2(x))_{g(x)}+q_2(x)(f_1(x))_{g(x)})g(x)+(f_1(x))_{g(x)}(f_2(x))_{g(x)})_{g(x)}\\
    &=((f_1(x))_{g(x)}(f_2(x))_{g(x)})_g(x)
  \end{align*}
\end{proof}
\vskip 0.5cm
\textbf{该定理可推广到多个多项式相加和相乘的情况。我们将以上运算分
别叫做多项式$f_1(x),\ f_2(x)$对$g(x)$的模加和模乘运算。}
\subsection{因式与倍式}
\begin{definition}
  设$f(x)$为域$\mathbb{F}$上的多项式,如果$f(x)$的因式只有$c,\ cf(x)$,其中$c \in \mathbb{F}^{*}$,则$f(x)$称为域$\mathbb{F}$上的\textcolor{blue}{不可约多项式},否则称为\textcolor{blue}{可约多项式}。
\end{definition}
\begin{theorem}
  域$\mathbb{F}$的多项式$g(x)\mid f_1(x),\ g(x) \mid f_2(x)$,那么对于$\mathbb{F}$上的任意多项式$s(x),t(x)$
  \begin{equation*}
    g(x) \mid s(x)f_1(x)+t(x)f_2(x)
  \end{equation*}
\end{theorem}
\begin{example}
  设$f(x)=x^2+1$,则$f(x)$是实数域$\mathbb{R}$上的不可约多项式,
  也是域$\mathbb{Z}_3$上的不可约多项式,但$f(x)$是复数域$\mathbb{C}$上的可约多项式,
  在$\mathbb{C}$上$f(x)=(x+\sqrt{-1})(x-\sqrt{-1}),\ f(x)$也是域$\mathbb{Z}_2$上的可约多项式,
  在$\mathbb{Z}$上$f(x)=(x+1)^2$。
\end{example}
\subsection{辗转相除法}
\begin{theorem}
  设$r_0(x),\ r_1(x)$为域$\mathbb{F}$上的两个多项式,$r_1(x) \neq 0$,则可以得如下带余除法算式:
  \begin{align*}
    r_0(x) = q_1(x)r_1(x)+r_2(x),&\quad 0 \leq degr_2(x) < degr_1(x)\\
    r_1(x) = q_2(x)r_2(x)+r_3(x),&\quad 0 \leq degr_3(x) < degr_2(x)\\
    \cdots & \quad \cdots\\
    r_{n-2}(x)=q_{n-1}(x)r_{n-1}(x)+r_n(x),&\quad 0 \leq degr_n(x)<degr_{n-1}(x)\\
    r_{n-1}(x)=q_n(x)r_n(x)+r_{n+1}(x),& \quad r_{n+1}(x) = 0
  \end{align*}
  \begin{itemize}
    \item 经过若干步后,余式必然为0;
    \item 存在多项式$s(x),\ t(x) \in \mathbb{F}[x]$,使得$s(x)r_0(x)+t(x)r_1(x)=r_n(x)$;
    \item 设$r_n(x)$首项系数为c,则$(r_0(x),r_1(x))= c^{-1}r_n(x)$,且最高公因式是唯一存在的。
    \item 对于任意$d(x) \in \mathbb{F}[x]$,若$d(x) \mid r_0(x),\ d(x)\mid r_1(x)$,那么$d(x)\mid (r_0(x),r_1(x))$。
  \end{itemize}
\end{theorem}
\begin{exercise}
  在$\mathbb{Z}_3$中$f(x)=x^3+x^2+x+1,\ g(x)=2x^4+x^2$试求多项式$s(x),\ t(x)$,使得$s(x)f(x)+t(x)g(x)= gcd(f(x),g(x))$。
\end{exercise}
\section{多项式整除和唯一因式分解定理}
\subsection{多项式整除的性质}
\begin{theorem}
  设$f(x),\ g(x)$为域$\mathbb{F}$上两个不全为零的多项式,则对于任意$k(x) \in \mathbb{F}[x],\ (f(x)+g(x)k(x),\ g(x))=(f(x),\ g(x))$。
\end{theorem}
\begin{theorem}
  设$p(x),\ f_1(x),\ f_2(x)$为域$\mathbb{F}$上的多项式,且$p(x) \mid f_1(x)f_2(x)$,若$(p(x),\ f_1(x))=1$,则$p(x) \mid f_2(x)$。
\end{theorem}
\begin{theorem}
  设$p(x),\ f_1(x),\ f_2(x)$为域$\mathbb{F}$上的多项式,$p(x)$为域$\mathbb{F}$上的不可约多项式,且$p(x) \mid f_1(x)f_2(x)$,则$p(x)\mid f_1(x)$或者$p(x) \mid f_2(x)$。
\end{theorem}
\subsection{多项式的模逆}
\begin{conclusion}
  设$f(x),\ g(x)$为域$\mathbb{F}$上两个次数大于0的多项式,那么,存在唯一的一对多项式$s(x),\ t(x) \in \mathbb{F}[x]$,使得$s(x)f(x)+t(x)g(x)$,且$degs(x)<degg(x)-deg(f(x),g(x)),\ degt(x)<degf(x)-deg(f(x),g(x))$
\end{conclusion}
\begin{proof}
  \begin{align*}
    s(x)f(x)+t(x)g(X)&=(f(x),g(x))=d(x)
    \\s(x)\frac{f(x)}{d(x)}+&t(x)\frac{g(x)}{d(x)} = 1
    \\degs(x)<deg(\frac{g(x)}{d(x)})&=degg(x)-degd(x)
    \\degf(x)<deg(\frac{f(x)}{d(x)})&=degf(x)-degd(x)
  \end{align*}
\end{proof}
\begin{proof}
  $u(x)f(x)+v(x)g(x)=1 \Rightarrow degu(x)<degg(x).$
  \begin{align*}
    u(x)=q_1(x)g(x)+s(x), degs(x)<degg(x),
    \\v(x)=q_2(x)f(x)+t(x), degt(x)<degf(x),
  \end{align*}
  得到:
  \begin{equation*}
    (q_1(x)g(x)+s(x))f(x)+(q_2(x)f(x)+t(x))g(x)=1
  \end{equation*}
  所以,$(q_1(x)+q_2(x))f(x)g(x)=1-(s(x)f(x)+t(x)g(x))$
  \\两边次数做比较,
  \begin{align*}
    deg((q_1(x)+q_2(x))f(x)g(x))=deg(q_1(x)+q_2(x))+deg(f(x)g(x))
    \\deg(1-(s(x)f(x)+t(x)g(x)))<deg((f(x)g(x)))
  \end{align*}
  所以,$q_1(x)+q_2(x)=0$,于是
  \begin{equation*}
    s(x)f(x)+t(x)g(x)=1
  \end{equation*}
  所以得到$degs(x)<degg(x) \Rightarrow degu(x)<degg(x)$。
\end{proof}
\subsection{唯一因式分解定理}
\begin{theorem}
  设$f(x)$是域$\mathbb{F}$上的次数大于0的多项式,则$f(x)$可以唯一地表示为域$\mathbb{F}$上一些次数大于0的不可约多项式的乘积。特别地,设$f(x)$是首1多项式,且,
  \begin{equation*}
    f(x)=p_1(x)p_2(x)\cdots p_s(x)=q_1(x)q_2(x)\cdots q_t(x)
  \end{equation*}
  其中$p_1(x),\ p_2(x),\ \cdots ,\ p_s(x),\ q_1(x),\ q_2(x),\ \cdots \ q_t(x)$均为域$\mathbb{F}$上次数大于0的首1不可约多项式,则$s=t$,经过适当的调整可使$p_1(x)=q_1(x),\ p_2(x)= q_2(x),\ \cdots ,\ p_s(x) = q_s(x)$。
\end{theorem}
\begin{exercise}
  $f(x)$是n次多项式,$f(x)$可约的充要条件是存在次数小于或者等于$[\frac{n}{2}]$的首1不可约因式。试判断$f(x)=x^7+x+1 \in \mathbb{Z}_2[x]$是否可约?
\end{exercise}
\begin{exercise}
  试判断$f(x)=x^7+x+1 \in \mathbb{Q}[x]$是否可约?
\end{exercise}
\section{扩域的构造}
\begin{itemize}
  \item 任何有限域元素的个数为$p^n,\ n \geq 1$,p为素数
  \item 任意$p^n$的有限域都存在
  \item 元素个数相同的有限域是同构的
\end{itemize}
\subsection{多项式的根}
\begin{definition}
  设$f(x)$为域$\mathbb{F}$上的多项式,如果$a \in \mathbb{F}$使得$f(a)=0$,则称$a$为$f(x)$在域$\mathbb{F}$中的一个根。
\end{definition}
\begin{theorem}[余元定理]
  设$f(x)$为域$\mathbb{F}$上的多项式,对于任意$a \in \mathbb{F}$,存在$g(x)\in \mathbb{F}[x]$使得
  \begin{equation*}
    f(x)=(x-a)g(x)+f(a)
  \end{equation*}
\end{theorem}
\begin{proof}
  不妨设$f(x)=(x-a)g(x)+c$,则$f(a)=(a-a)g(a)+c$,所以$c = f(a)$。
\end{proof}
\vskip 0.5cm
\begin{conclusion}
  设$f(x)$为域$\mathbb{F}$上的多项式,a为$f(x)$在域$\mathbb{F}$中的根的充要条件是$(x-a)\mid f(x)$。
\end{conclusion}
\vskip 0.5cm
\begin{conclusion}
  设$f(x)$为域$\mathbb{F}$上的$n \geq 1$次多项式,如果$a_1,\ a_2,\ \cdots ,\ a_m$为$f(x)$在域$\mathbb{F}$中的$m \geq 1 $个不同的根,则存在$n-m$次多项式$g(x) \in \mathbb{F}[x]$使得,
  \begin{equation*}
    f(x) = (x-a_1)(x-a_2)\cdots(x-a_m)g(x)
  \end{equation*}
\end{conclusion}
\vskip 0.5cm
\begin{conclusion}
  设$f(x)$为域$\mathbb{F}$上的$n \geq 1$次多项式,则$f(x)$在$\mathbb{F}$的任意扩域中,不同根的个数都不会超过$n$。
\end{conclusion}
\subsection{扩域的构造}
\begin{theorem}
  设$f(x)$为域$\mathbb{F}$上的$n \geq 1$次不可约多项式
\end{theorem}
\section{有限域的乘法群}
\subsection{有限域的乘法群}
\subsection{循环群的结构}
\begin{theorem}
  设$<a>$是由$a$生成的$n$阶的循环群,试证明:
  \begin{enumerate}[(1)]
    \item $<a>$的子群都是循环群;
    \item 对于任意正整数$a \mid n,\ <a>$存在唯一的$d$元子群;
    \item 若整数$s,\ t $不全为$0$,则$<a^s,\ a^t>=<a^{(s,t)}>$。
  \end{enumerate}
\end{theorem}
\begin{proof}
  \begin{enumerate}[(1)]
    \item 设G是$<a>$任一子群,如果G是单位元群,结论成立。\\ 否则G由$a^i,\ 0 \leq i < n$中的一部分组成,可设G中a的最小正幂为$a^m$,
    则根据G中乘法的封闭性,那么$<a^m>$是G的子群,且为循环群。\\
    下面证明$G = <a^m>$\\
    对于$\forall a^k \in G,\ k \in \mathbb{Z}$,设$k = qm+r,\ 0 \leq r <m$
    \begin{align*}
      &a^r = a^k(a^{-m})^q \in G \Rightarrow r =0 ,
      \\&m \mid k,\ a^k = a^{mq} \in <a^m>,\ G \subseteq <a^m>.
    \end{align*}
    由此说明:$<a>$的子群G是由G中a的最小正幂$a^m$生成的循环群。
    \item 存在性:对于$d \mid n$,令$m = \frac{n}{d}$,因为$ord(a^m)= \frac{n}{(m,n)} = \frac{n}{m}= d$,所以群$<a^m>$即为$d$阶子群。
    \\唯一性:设G是$<a>$的任一$d$阶子群,该群中a的最小正幂为$a^m$,由(1)的证明,$G = <a^m>$,又因为$a^n =e \in G$,所以,由(1),$m \mid n$
    因此$a^m$的阶为$\frac{n}{m} = d$,所以$m = \frac{n}{d}$。
    \\由此说明,$<a>$的所有d阶子群均由同一$a^m$生成,它们是唯一的。
    \item $<a^s,\ a^t>= \{a^{sx+ty}\mid x,\ y \in \mathbb{Z} \}$,
    \\
    \end{enumerate}
\end{proof}
\begin{example}
  试求域$\mathbb{Z}_2[x]_{x^6+x+1}$中包含多项式$x^2+x+1$的根的最小子域。
\end{example}
\begin{solution}
  
\end{solution}
\chapter{}
\section{勒让德符号}
\begin{definition}
  设𝑚是正整数,若同余式
  \begin{equation*}
    x^2 \equiv a(mod\ m),\ (a,m) =1
  \end{equation*}
  有解,则a称作模m的二次剩余(或平方剩余),否则,a称作模𝑚的二次非
剩余(或平方非剩余)
\begin{enumerate}
  \item 模m的同余式可以转化成模p的同余式
\end{enumerate}
\end{definition}
\begin{definition}
  设𝑝是素数,定义勒让德(Legendre)符号如下:
\end{definition}
\begin{example}
  对于素数 7,因 为 $(±1)^2 \equiv 1, (±2)^2 \equiv 4, (±3)^2 \equiv 2(mod\ 7)$,所以\\
  \begin{align*}
    (\frac{1}{7})
  \end{align*}
\end{example}
\subsection{欧拉判别法则}
设𝑝是素数,求同余式$x^2 \equiv a(mod\ p)$的解可以看成是在有限域$\mathbb{Z}_p$中求多项式$x^2 − a$的根。
\begin{theorem}[欧拉判别法则]
设$p$是奇素数,则对于任意整数$a$,$(\frac{a}{p}) \equiv a^{\frac{p-1}{2}}(mod\ p)$。

\end{theorem}
\end{document}
